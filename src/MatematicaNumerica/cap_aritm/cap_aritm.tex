%Este trabalho está licenciado sob a Licença Atribuição-CompartilhaIgual 4.0 Internacional Creative Commons. Para visualizar uma cópia desta licença, visite http://creativecommons.org/licenses/by-sa/4.0/deed.pt_BR ou mande uma carta para Creative Commons, PO Box 1866, Mountain View, CA 94042, USA.

\chapter{Aritmética de Máquina}\label{cap_aritm}
\thispagestyle{fancy}

\ifisoctave
No \verb+GNU Octave+, temos
\begin{verbatim}
>> 0.1+0.2==0.3
ans = 0
\end{verbatim}
\fi

<<<<<<< HEAD
\section{Sistema de numeração posicional}\label{cap_aritm_sec_sisnumpos}
=======
\section{Sistema de numeração posicional}\label{cap_artm_sec_sisnumpos}
>>>>>>> dcc7491fb25d0fccd958b0f26b35c9115c81f65d

Cotidianamente, usamos o sistema de numeração posicional na base decimal. Por exemplo, temos
\begin{equation}
  123,5 = 1\times 10^2 + 2\times 10^1 + 3\times 10^0 + 5\times 10^{-1},
\end{equation}
onde o algarismo/dígito 1 está na posição 2 (posição das centenas), o dígito 2 está na posição 1 (posição das dezenas) e o dígito 3 está na posição 0 (posição das unidades). Mais geralmente, temos a representação decimal
\begin{align}
  \pm &d_n\ldots d_2d_1d_0,d_{-1}d_{-2}d_{-3}\ldots := \\
  \pm &\left(d_n\times 10^n + \cdots + d_2\times 10^2 + d_1\times 10^1 + d_0\times 10^0\right. \\
      &\left. + d_{-1}\times 10^{-1} + d_{-2}\times 10^{-2} + d_{-3}\times 10^{-3} + \cdots\right),
\end{align}
cujos os dígitos $d_i \in \{0, 1, 2, 3, 4, 5, 6, 7, 8, 9\}$, $i=n, \dotsc, 2, 1, 0, -1, -2, -3, \ldots$. Observamos que esta representação posicional pode ser imediatamente generalizada para outras bases numéricas.

\begin{defn}\normalfont{(Representação posicional)}\label{defn:representacao_posicional}
  Dada uma base $b\in\mathbb{N}\setminus \{0\}$, definimos a representação
  \begin{align}
    \pm &(d_n\ldots d_2d_1d_0,d_{-1}d_{-2}d_{-3}\ldots)_{b} := \\
    \pm &\left(d_n\times b^n + \cdots + d_2\times b^2 + d_1\times b^1 + d_0\times b^0\right. \\
      &\left. + d_{-1}\times b^{-1} + d_{-2}\times b^{-2} + d_{-3}\times b^{-3} + \cdots\right),
  \end{align}
onde os dígitos $d_i\in\{0, 1, \dotsc, b-1\}$\footnote{Para bases $b\geq 11$, usamos a representação dos dígitos maiores ou iguais a 10 por letras maiúsculas do alfabeto latino, i.e. $A=10$, $B=11$, $C=12$ e assim por diante.}, $i=n, \dotsc, 2, 1, 0, -1, -2, -3, \ldots$.
\end{defn}

\begin{ex}\normalfont{(Representação binária)}\label{ex:base_binaria}
  O número $(11010,101)_2$ está escrito na representação binária (base $b=2$). Da Definição~\ref{defn:representacao_posicional}, temos
  \begin{align}
    (\stackrel{4}{1}~\stackrel{3}{1}~\stackrel{2}{0}~\stackrel{1}{1}~\stackrel{0}{0},\stackrel{-1}{~\,1}~\stackrel{-2}{~\,0}~\stackrel{-3}{~\,1})_2 &= 1\times 2^4 + 1\times 2^3 + 0\times 2^2 + 1\times 2^1 + 0\times 2^0\\
    &+ 1\times 2^{-1} + 0\times 2^{-2} + 1\times 2^{-3}\\
    &= 26,625.
  \end{align}
\ifisoctave
Podemos fazer estas contas no \verb+GNU Octave+ da seguinte forma
\begin{verbatim}
>> 1*2^4+1*2^3+0*2^2+1*2^1+0*2^0+1*2^-1+0*2^-2+1*2^-3
ans =  26.625
\end{verbatim}
\fi
\end{ex}

\subsection{Mudança de base}

Um mesmo número pode ser representado em diferentes bases e, aqui, estudaremos como obter a representação de uma número em diferentes bases. A mudança de base de representação de um dado número pode ser feita de várias formas. De forma geral, se temos um número $x$ representado na base $b_1$ e queremos obter sua representação na base $b_2$, fazemos
\begin{enumerate}[1.]
\item Calculamos a representação do número $x$ na base decimal.
\item Da calculada representação decimal, calculamos a representação de $x$ na base $b_2$.
\end{enumerate}
Observamos que o passo 1. ($b \to 10$) segue imediatamente da Definição \ref{defn:representacao_posicional}. Agora, o passo 2. ($10\to b$), podemos usar o seguinte procedimento. Suponhamos que $x$ tenha a seguinte representação decimal
\begin{equation}
  d_nd_{n-1}d_{n-2}\ldots d_0,d_{-1}d_{-2}d_{-3}\ldots
\end{equation}
Então, separamos sua parte inteira $I = d_nd_{n-1}d_{n-2}\ldots d_0$ e sua parte fracionária $F = 0,d_{-1}d_{-2}d_{-3}\ldots$ ($x = I + F$). Então, usando de sucessivas divisões de $I$ pela base $b$ desejada, obtemos sua representação nesta mesma base. Analogamente, usando de sucessivas multiplicações de $F$ pela base $b$, obtemos sua representação nesta base. Por fim, basta somar as representações calculadas.

\begin{ex}
  Obtenha a representação em base quartenária ($b=4$) do número $(11010,101)_2$.
  \begin{itemize}
  \item[1.] $b=2 \to b=10$. 
A representação de $(11010,101)_2$ segue direto da Definição \ref{defn:representacao_posicional} (veja, o Exemplo~\ref{ex:base_binaria}). Ou seja, temos
\begin{align}
    (\stackrel{4}{1}~\stackrel{3}{1}~\stackrel{2}{0}~\stackrel{1}{1}~\stackrel{0}{0},\stackrel{-1}{~\,1}~\stackrel{-2}{~\,0}~\stackrel{-3}{~\,1})_2 &= 2^4 + 2^3 + 2^1 + 2^{-1} + 2^{-3} \\
  &= 26,625.
\end{align}
  \end{itemize}
\ifisoctave
No \verb+GNU Octave+ podemos fazer a mudança para a base decimal com a função \verb+base2dec+:
\begin{verbatim}
>> I = base2dec("11010",2)
I =  26
>> F = base2dec("101",2)*2^-3
F =  0.62500
>> I+F
ans =  26.625
\end{verbatim}
\fi

\begin{itemize}
\item[2.] $b=10 \to b=4$.
\end{itemize}
Primeiramente, decompomos $26,625$ em sua parte inteira $I = 26$ e em sua parte fracionária $0,625$. Então, ao fazermos sucessivas divisões de $I$ por $b=4$, obtemos:
\begin{align}
  I &= 26\\
  &= 6\times 4 + 2\times 4^0\\
  &= (1\times 4 + 2)\times 4 + 2\times 4^0\\
  &= 1\times 4^2 + 2\times 4 + 2\times 4^0\\
  &= (122)_4.
\end{align}
Agora, para a parte fracionária, usamos sucessivas multiplicações de $F$ por $b=4$, obtendo:
\begin{align}
  F &= 0,625\\
  &= 2,5\times 4^{-1} = 2\times 4^{-1} + 0,5\times 4^{-1}\\
  &= 2\times 4^{-1} + 2\times 4^{-1}\times 4^{-1}\\
  &= 2\times 4^{-1} + 2\times 4^{-2}\\
  &= (0,22)_{4}.
\end{align}
No \verb+GNU Octave+, podemos computar a representação de $F$ na base $b=4$ da seguinte forma:
\begin{verbatim}
>> F=0.625
F =  0.62500
>> d=fix(F*4),F=F*4-d
d =  2
F =  0.50000
>> d=fix(F*4),F=F*4-d
d =  2
F = 0
\end{verbatim}

Por fim, dos passos 1. e 2., temos $(11010,101)_2 = (122,22)_4$.
\end{ex}

\subsection{Exercícios}

\begin{exer}\label{exer:base2dec}
  Obtenha a representação decimal dos seguinte números:
  \begin{enumerate}[a)]
  \item $(101101,00101)_2$
  \item $(23,1)_4$
  \item $(DAAD)_{16}$
  \item $(0,1)_3$
  \item $(0,\overline{1})_4$
  \end{enumerate}
\end{exer}
\begin{resp}
    \ifisoctave 
  \href{https://github.com/phkonzen/notas/blob/master/src/MatematicaNumerica/cap_aritm/dados/exer_base2dec/exer_base2dec.m}{Código.} 
  \fi
  a)~$89,15625$; b)~$11,25$; c)~$55981$; d)~$0,\overline{3}$; e)~$0,\overline{3}$
\end{resp}

\begin{exer}\label{exer_dec2base}
  Obtenha a representação dos seguintes números na base indicada:
  \begin{enumerate}[a)]
  \item $45,5$ na base $b=2$.
  \item $0,3$ na base $b=4$.
  \end{enumerate}
\end{exer}
\begin{resp}
  \ifisoctave 
  \href{https://github.com/phkonzen/notas/blob/master/src/MatematicaNumerica/cap_aritm/dados/exer_dec2base/exer_dec2base.m}{Código.} 
  \fi
  a)~$(101101,1)_2$; b)~$(0,1\overline{03})_4$
\end{resp}

\begin{exer}\label{exer_base2base}
  Obtenha a representação dos seguintes números na base indicada:
  \begin{enumerate}[a)]
  \item $(101101,00101)_2$ na base $b=4$.
  \item $(23,1)_4$ na base $b=2$.
  \end{enumerate}
\end{exer}
\begin{resp}
  \ifisoctave 
  \href{https://github.com/phkonzen/notas/blob/master/src/MatematicaNumerica/cap_aritm/dados/exer_base2base/exer_base2base.m}{Código.} 
  \fi
  a)~$(1121,022)_4$; b)~$(1011,01)_2$  
\end{resp}

<<<<<<< HEAD
\section{Representação de números em máquina}\label{cap_aritm_sec_nummaq}
=======
\section{Representação de números em máquina}\label{cap_artm_sec_repummaq}
>>>>>>> dcc7491fb25d0fccd958b0f26b35c9115c81f65d

\emconstrucao