%Este trabalho está licenciado sob a Licença Atribuição-CompartilhaIgual 4.0 Internacional Creative Commons. Para visualizar uma cópia desta licença, visite http://creativecommons.org/licenses/by-sa/4.0/deed.pt_BR ou mande uma carta para Creative Commons, PO Box 1866, Mountain View, CA 94042, USA.

\chapter{Integração}\label{cap_integr}
\thispagestyle{fancy}

Neste capítulo, discutimos os métodos numéricos fundamentais para a aproximação de integrais definidas de funções. Tais métodos são chamados de \emph{quadraturas numéricas} e têm a forma
\begin{equation}
  \int_a^b f(x)\,dx \approx \sum_{i=1}^n f(x_i)w_i,
\end{equation}
onde $x_i$ e $w_i$ são, respectivamente, o $i$-ésimo nodo e o $i$-ésimo peso da quadratura, $i=1, 2, \dotsc, n$.

\section{Regras de Newton-Cotes}\label{cap_integr_sec_NC}

Dada uma função $f(x)$ e um intervalo $[a, b]$, denotamos por
\begin{equation}
  I := \int_a^b f(x)\,dx.
\end{equation}
a integral de $f(x)$ no intervalo $[a, b]$. A ideia das regras de Newton-Cotes e aproximar $I$ pela integral de um polinômio interpolador de $f(x)$ por pontos previamente selecionados.

Seja, então, $p(x)$ o polinômio interpolador de grau $n$ de $f(x)$ pelos dados pontos $\{(x_i, f(x_i))\}_{i=1}^{n+1}$, com $x_1 < x_2 < \cdots < x_{n+1}$ e $x_i\in [a, b]$ para todo $i=1, 2, \dotsc, n+1$. Então, pelo teorema de Lagrange, temos
\begin{equation}
  f(x) = p(x) + R_{n+1}(x),
\end{equation}
onde
\begin{equation}
  p(x) = \sum_{i=1}^{n+1} f(x_i)\prod_{\overset{j=1}{j\neq i}}^{n+1} \frac{(x-x_j)}{x_i-x_j}
\end{equation}
e
\begin{equation}
  R_{n+1}(x) = \frac{f^{(n+1)}(\xi)}{(n+1)!}\prod_{j=1}^{n+1}(x-x_j),
\end{equation}
onde $\xi = \xi(x)$ pertencente ao intervalo $[x_1, x_{n+1}]$. Deste modo, temos
\begin{align}
  I &:= \int_a^b f(x)\\
  &= \int_a^b p(x)\,dx + \int_a^b R_{n+1}(x)\,dx\\
  &= \underbrace{\sum_{i=1}^{n+1} f(x_i)\int_a^b \prod_{\overset{j=1}{j\neq i}}^{n+1} \frac{(x-x_j)}{x_i-x_j)}\,dx}_{\text{quadratura}} + \underbrace{\int_a^b R_{n+1}(x)\,dx}_{\text{erro de truncamento}}
\end{align}
Ou seja, nas quadraturas (regras) de Newton-Cotes, os nodos são as abscissas dos pontos interpolados e os pesos são as integrais dos polinômios de Lagrange associados.

Na sequência, abordaremos as regras de Newton-Cotes mais usuais e estimaremos o erro de truncamento caso a caso. Para uma abordagem mais geral, recomenda-se consultar~\cite[Ch 7.,Sec. 1.1]{Isaacson1994a}.

\subsection{Regras de Newton-Cotes fechadas}

As regras de Newton-Cotes fechadas são aqueles que a quadratura incluem os extremos do intervalo de integração, i.e. os nodos extremos são $x_1=a$ e $x_{n+1}=b$.

\subsubsection{Regra do trapézio}

A regra do trapézio é obtida tomando-se os nodos $x_1=a$ e $x_2=b$. Então, denotando $h:=b-a$\footnote{Neste capítulo, $h$ é escolhido como a distância entre os nodos.}, os pesos da quadratura são:
\begin{align}
  w_1 &= \int_a^b \frac{x-b}{a-b}\,dx \\
  &= \frac{(b-a)}{2} = \frac{h}{2}
\end{align}
e
\begin{align}
  w_2 &= \int_a^b \frac{x-a}{b-a}\,dx \\
  &= \frac{(b-a)}{2} = \frac{h}{2}.
\end{align}
Agora, estimamos o erro de truncamento com
\begin{align}
  E &:= \int_a^b R_2(x)\,dx\\
  &= \int_a^b \frac{f''(\xi(x))}{2}(x-a)(x-b)\,dx\\
  &\leq C\left|\int_a^b (x-a)(x-b)\,dx\right|\\
  &= C\frac{(b-a)^3}{6} = O(h^3).
\end{align}

Portanto, a \emph{regra do trapézio}\index{regra do!trapézio} é dada por
\begin{equation}
  \int_a^b f(x)\,dx = \frac{h}{2}(f(a) + f(b)) + O(h^3).
\end{equation}

\begin{ex}\label{ex:int_trap}
  Consideremos o problema de computar a integral de $f(x)=xe^{-x^2}$ no intervalo $[0, 1/4]$. Analiticamente, temos
  \begin{align}
    I = \int_0^{1/4} xe^{-x^2}\,dx &= \left. -\frac{e^{-x^2}}{2} \right|_0^{1/4}\\
    &= \frac{1-e^{-1/4}}{2} = 3,02935\E-2.
  \end{align}
Agora, usando a regra do trapézio, obtemos a seguinte aproximação para $I$
\begin{align}
  I &\approx \frac{h}{2}(f(0) + f(1/2))\\
  &= \frac{1/4}{2}\left(0 + \frac{1}{4}e^{-(1/4)^2}\right) = 2,93567\E-2.
\end{align}

\ifisoctave
Podemos obter a aproximação dada pela regra do trapézio no \verb+GNU Octave+ com o seguinte código:
\begin{verbatim}
f = @(x) x*exp(-x^2);
a=0;
b=0.25;
h=b-a;
Itrap = (h/2)*(f(a)+f(b));
printf("%1.5E\n",Itrap)
\end{verbatim}
\fi
\end{ex}

\subsubsection{Regra de Simpson}

A regra de Simpson é obtida escolhendo-se os nodos $x_1=a$, $x_2=(a+b)/2$ e $x_3=b$. Com isso e denotando $h=(b-a)/2$, calculamos os seguintes pesos:
\begin{align}
  w_1 &= \int_a^b\frac{(x-x_2)(x-x_3)}{(x_1-x_2)(x_1-x_3)}\,dx\\
  &= \frac{(b-a)}{6} = \frac{h}{6},
\end{align}
\begin{align}
  w_2 &= \int_a^b\frac{(x-x_1)(x-x_3)}{(x_2-x_1)(x_2-x_3)}\,dx\\
  &= 4\frac{(b-a)}{6} = 4\frac{h}{6}
\end{align}
e
\begin{align}
  w_3 &= \int_a^b\frac{(x-x_1)(x-x_2)}{(x_3-x_1)(x_3-x_2)}\,dx\\
  &= \frac{(b-a)}{6} = \frac{h}{6}.
\end{align}
Isto nos fornece a chamada \emph{regra de Simpson}\index{regra de Simpson}
\begin{equation}\label{eq:aux_Simpson}
  I \approx \frac{h}{6}\left[f(a) + 4f\left(\frac{a+b}{2}\right) + f(b)\right]
\end{equation}

Nos resta estimar o erro de truncamento da regra de Simpson. Para tanto, consideramos a expansão em polinômio de Taylor de grau 3 de $f(x)$ em torno do ponto $x_2$, i.e.
\begin{align}
  f(x) &= f(x_2) + f'(x_2)(x-x_2) + \frac{f''(x_2)}{2}(x-x_2)^2 \nonumber\\
  &+ \frac{f'''(x_2)}{6}(x-x_2)^3 \nonumber\\
  &+ \frac{f^{(4)}(\xi_1(x))}{24}(x-x_2)^4,
\end{align}
donde
\begin{align}
  \int_a^b f(x)\,dx &= 2hf(x_2) + \frac{h^3}{3}f''(x_2) \nonumber\\
  &+ \frac{1}{24}\int_a^bf^{(4)}(\xi_1(x))(x-x_2)^4\,dx.\label{eq:aux_int_sim1}
\end{align}
Daí, usando da fórmula de diferenças finitas central de ordem $h^2$, temos
\begin{equation}\label{eq:aux_int_sim2}
  f''(x_2) = \frac{f(x_1) - 2f(x_2) + f(x_3)}{h^2} + O(h^2).
\end{equation}
Ainda, o último termo da equação~\eqref{eq:aux_int_sim1} pode ser estimado por
\begin{align}
  \left|\frac{1}{24}\int_a^bf^{(4)}(\xi_1(x))(x-x_2)^4\,dx\right| &\leq C\left|\int_a^b (x-x_2)^4\,dx\right|\\
  &= C(b-a)^5 = O(h^5).\label{eq:aux_int_sim3}
\end{align}\label{eq:aux_int_sim3}
Então, de \eqref{eq:aux_int_sim1}, \eqref{eq:aux_int_sim2} e \eqref{eq:aux_int_sim3}, temos
\begin{equation}
  \int_a^b f(x)\,dx = \frac{h}{3}\left[f(a) + 4f\left(\frac{a+b}{2}\right) + f(b)\right] + O(h^5),
\end{equation}
o que mostra que a \emph{regra de Simpson tem erro de truncamento da ordem $h^5$}.

\begin{ex}\label{ex:int_simp}
  Aproximando a integral dada no Exemplo~\ref{ex:int_trap} pela a regra de Simpson, temos
  \begin{align}
    \int_0^{1/4} f(x)\,dx &\approx \frac{1/8}{3}\left[f(0) + 4f\left(\frac{1}{8}\right) + f\left(\frac{1}{4}\right)\right]\\
    &= \frac{1}{24}\left[\frac{1}{2}e^{-(1/8)^2} + \frac{1}{4}e^{-(1/4)^2}\right]\\
    &= 3,02959\E-2.
  \end{align}

\ifisoctave
Podemos computar a aproximação dada pela regra de Simpson no \verb+GNU Octave+ com o seguinte código:
\begin{verbatim}
f = @(x) x*exp(-x^2);
a=0;
b=1/4;
h=(b-a)/2;
Isimp = (h/3)*(f(a)+4*f((a+b)/2)+f(b));
printf("%1.5E\n",Isimp)
\end{verbatim}
\fi
\end{ex}

\subsection{Regras de Newton-Cotes abertas}

As regras de Newton-Cotes abertas não incluem os extremos dos intervalos como nodos das quadraturas.

\subsubsection{Regra do ponto médio}

A regra do ponto médio\index{regra do!ponto médio} é obtida usando apenas o nodo $x_1=(a+b)/2$. Desta forma, temos
\begin{equation}
  \int_a^b f(x)\,dx = \int_a^b f(x_1)\,dx + \int_a^b f'(\xi(x))(x-x_1)\,dx,
\end{equation}
donde, denotando $h:=(b-a)$, temos
\begin{equation}
  \int_a^b f(x),dx = hf\left(\frac{a+b}{2}\right) + O(h^3).
\end{equation}
Deixa-se para o leitor a verificação do erro de truncamento (veja, Exercício~\ref{exer:trunc_pto_medio}).

\begin{ex}\label{ex:int_pto_medio}
  Aproximando a integral dada no Exemplo~\ref{ex:int_trap} pela a regra do ponto médio, temos
  \begin{align}
    \int_0^{1/4} f(x)\,dx &\approx \frac{1}{4}f\left(\frac{1}{8}\right)\\
    &= \frac{1}{32}e^{-(1/8)^2}\\
    &= 3,07655\E-2
  \end{align}

\ifisoctave
Podemos computar a aproximação dada pela regra do ponto médio no \verb+GNU Octave+ com o seguinte código:
\begin{verbatim}
f = @(x) x*exp(-x^2);
a=0;
b=0.25;
h=b-a;
Ipmd = h*f((a+b)/2);
printf("%1.5E\n",Ipmd)
\end{verbatim}
\fi
\end{ex}

\subsection*{Exercício}

\begin{exer}\label{exer:int_NC_fun}
  Aproxime
  \begin{equation}
    \int_{-1}^0 \frac{\sen(x+2)-e^{-x^2}}{x^2+\ln(x+2)}\,dx
  \end{equation}
usando a:
\begin{enumerate}[a)]
\item regra do ponto médio.
\item regra do trapézio.
\item regra de Simpson.
\end{enumerate}
\end{exer}
\begin{resp}
  \ifisoctave 
  \href{https://github.com/phkonzen/notas/blob/master/src/MatematicaNumerica/cap_integr/dados/exer_int_NC_fun/exer_int_NC_fun.m}{Código.} 
  \fi
  a)~$3,33647\E-1$; b)~$1,71368\E-1$; c)~$2,79554\E-1$
\end{resp}

\begin{exer}\label{exer:int_NC_tab}
  Considere a seguinte tabela de pontos
  \begin{center}
    \begin{tabular}{l|cccccc}
      $i$ & $1$ & $2$ & $3$ & $4$ & $5$ & $6$ \\\hline
      $x_i$ & $2,0$ & $2,1$ & $2,2$ & $2,3$ & $2,4$ & $2,5$ \\
      $y_i$ & $1,86$ & $1,90$ & $2,01$ & $2,16$ & $2,23$ & $2,31$ \\\hline
    \end{tabular}
  \end{center}
Assumindo que $y = f(x)$, calcule:
\begin{enumerate}[a)]
\item $\displaystyle \int_{2,1}^{2,3} f(x)\,dx$ usando a regra do ponto médio.
\item $\displaystyle \int_{2,0}^{2,5} f(x)\,dx$ usando a regra do trapézio.
\item $\displaystyle \int_{2,0}^{2,4} f(x)\,dx$ usando a regra de Simpson.
\end{enumerate}
\end{exer}
\begin{resp}
  \ifisoctave 
  \href{https://github.com/phkonzen/notas/blob/master/src/MatematicaNumerica/cap_integr/dados/exer_int_NC_tab/exer_int_NC_tab.m}{Código.} 
  \fi
  a)~$4,02000\E-1$; b)~$1,04250E+0$; c)~$8,08667\E-1$
\end{resp}

\begin{exer}\label{exer:trunc_pto_medio}
  Mostre que o erro de truncamento da regra do ponto médio é da ordem de $h^3$, onde $h$ é o tamanho do intervalo de integração.
\end{exer}
\begin{resp}
  Use um procedimento semelhante aquele usado para determinar a ordem do erro de truncamento da regra de Simpson.
\end{resp}

\begin{exer}\label{exer:NC_aberta_2pts}
  Obtenha a regra de Newton-Cotes aberta de $2$ pontos e estime seu erro de truncamento.
\end{exer}
\begin{resp}
  \begin{align}
    \displaystyle \int_a^bf(x)\,dx &= \frac{3h}{2}\left[f\left(a+\frac{1}{3}(b-a)\right)\right. \\
    &+ \left. f\left(a + \frac{2}{3}(b-a)\right)\right] + O(h^3), ~h=\frac{(b-a)}{3}
  \end{align}
\end{resp}

\section{Regras compostas de Newton-Cotes}\label{cap_integr_sec_int_comp}

Regras de integração numérica compostas (ou quadraturas compostas\index{quadratura composta}) são aquelas obtidas da composição de quadraturas aplicadas as subintervalos do intervalo de integração. Mais especificamente, a integral de uma dada função $f(x)$ em um dado intervalo $[a, b]$ pode ser reescrita como uma soma de integrais em sucessivos subintervalos de $[a, b]$, i.e.
\begin{equation}
  \int_a^b f(x)\,dx = \sum_{i=1}^{n} \int_{x_i}^{x_{i+1}}f(x)\,dx,
\end{equation}
onde $a=x_1 < x_2 < \cdots < x_{n+1}=b$. Então, a aplicação de uma quadratura em cada integral em $[x_i, x_{i+1}]$, $i=1, 2, \dotsc, n$, nos fornece uma regra composta.

\subsection{Regra composta do ponto médio}

Consideremos uma partição uniforme do intervalo de integração $[a, b]$ da forma $a=\tilde{x}_1 < \tilde{x}_2 < \cdots < \tilde{x}_{n+1}=b$, com $h=x_{i+1}-x_{i}$, $i=1, 2, \dotsc, n$. Então, aplicando a regra do ponto médio a cada integral nos subintervalos $[\tilde{x}_i, \tilde{x}_{i+1}]$, temos
\begin{align}
  \int_a^b f(x)\,dx &= \sum_{i=1}^{n}\int_{\tilde{x}_i}^{\tilde{x}_{i+1}}f(x)\,dx\\
  &= \sum_{i=1}^n \left[hf\left(\frac{\tilde{x}_i+\tilde{x}_{i+1}}{2}\right) + O(h^3)\right].
\end{align}
Agora, observando que $h:=(b-a)/n$ e escolhendo os nodos $x_i = a + (i-1/2)h$, $i=1, 2, \dotsc, n$, obtemos a \emph{regra composta do ponto médio com $n$ subintervalos}
\begin{equation}
  \int_a^b f(x)\,dx = \sum_{i=1}^n hf(x_i) + O(h^2).
\end{equation}

\begin{ex}\label{ex:int_comp_pm}
  Consideremos o problema de computar a integral de $f(x)=xe^{-x^2}$ no intervalo $[0, 1]$. Usando a regra composta do ponto médio com $n$ subintervalos, obtemos a aproximação
  \begin{equation}
    \underbrace{\int_a^b f(x)\,dx}_{I} \approx \underbrace{\sum_{i=1}^n hf(x_i)}_{S},
  \end{equation}
onde $h=1/(4n)$ e $x_i = (i-1/2)h$, $i=1, 2, \dotsc, n$. Na Tabela~\ref{tab:ex_int_comp_pm}, temos as aproximações computadas com diversos números de subintervalos, bem como, seus erros absolutos.

\begin{table}[h!]
  \centering
  \caption{Resultados referentes ao Exemplo~\ref{ex:int_comp_pm}.}
  \begin{tabular}{l|cc}
    $n$ & $S$ & $|I-S|$ \\\hline
    1   & $3,89400\E-1$ & $7,3\E-2$ \\
    10  & $3,16631\E-1$ & $5,7\E-4$ \\
    100 & $3,16066\E-1$ & $5,7\E-6$ \\
    1000& $3.16060\E-1$ & $5,7\E-8$ \\\hline
  \end{tabular}
  \label{tab:ex_int_comp_pm}
\end{table}

\ifisoctave
Podemos fazer estas computações com o auxílio do seguinte código \verb+GNU Octave+:
\begin{verbatim}
f = @(x) x*exp(-x^2);
a=0;
b=1;
n=10;
h=(b-a)/n;
s=0;
for i=1:n
  x=a+(i-1/2)*h;
  s+=h*f(x);
endfor
printf("%1.5E %1.1E\n",s,abs((1-e^(-1))/2-s))
\end{verbatim}
\fi
\end{ex}

\subsection{Regra composta do trapézio}

Para obtermos a regra composta do trapézio, consideramos uma partição uniforme do intervalo de integração $[a, b]$ da forma $a=x_1 < x_2 < \cdots < x_{n+1}=b$ com $h=x_{i+1}-x_{i}$, $i=1, 2, \dotsc, n$. Então, aplicando a regra do trapézio em cada integração nos subintervalos, obtemos
\begin{align}
  \int_a^bf(x)\,dx &= \sum_{i=1}^n \int_{x_i}^{x_{i+1}} f(x)\,dx\\
  &= \sum_{i=1}^n \left\{\frac{h}{2}\left[f(x_i)+f(x_{i+1})\right] + O(h^3)\right\}\\
  &= f(x_1)\frac{h}{2} + \sum_{i=2}^{n} hf(x_i) + f(x_{n+1})\frac{h}{2} + O(h^2). 
\end{align}
Desta forma, a regra composto do trapézio\index{regra composta!do trapézio} com $n$ subintervalos é
\begin{equation}
  \int_a^b f(x)\,dx = \frac{h}{2}\left[f(x_1) + \sum_{i=2}^{n} 2f(x_i) + f(x_{n+1})\right] + O(h^2),
\end{equation}
onde $h=(b-a)/n$ e $x_i = a + (i-1)h$, $i=1, 2, \dotsc, n$.

\begin{ex}\label{ex:int_comp_trap}
  Consideremos o problema de computar a integral de $f(x)=xe^{-x^2}$ no intervalo $[0, 1]$. Usando a regra composta do trapézio com $n$ subintervalos, obtemos a aproximação
  \begin{equation}
    \underbrace{\int_a^b f(x)\,dx}_{I} \approx \underbrace{\frac{h}{2}\left[f(x_1) + 2\sum_{i=2}^{n-1} f(x_i) + f(x_{n+1})\right]}_{S},
  \end{equation}
onde $h=1/(4n)$ e $x_i = (i-1)h$, $i=1, 2, \dotsc, n$. Na Tabela~\ref{tab:ex_int_comp_trap}, temos as aproximações computadas com diversos números de subintervalos, bem como, seus erros absolutos.

\begin{table}[h!]
  \centering
  \caption{Resultados referentes ao Exemplo~\ref{ex:int_comp_trap}.}
  \begin{tabular}{l|cc}
    $n$ & $S$ & $|I-S|$ \\\hline
    1   & $1,83940\E-1$ & $1,3\E-1$ \\
    10  & $3,14919\E-1$ & $1,1\E-3$ \\
    100 & $3.16049\E-1$ & $1,1\E-5$ \\
    1000& $3,16060\E-1$ & $1,1\E-7$ \\\hline
  \end{tabular}
  \label{tab:ex_int_comp_trap}
\end{table}

\ifisoctave
Podemos fazer estas computações com o auxílio do seguinte código \verb+GNU Octave+:
\begin{verbatim}
f = @(x) x*exp(-x^2);
a=0;
b=1;
n=1000;
h=(b-a)/n;
s=f(a);
for i=2:n
  x=a+(i-1)*h;
  s+=2*f(x);
endfor
s+=f(b);
s*=h/2;
printf("%1.5E %1.1E\n",s,abs((1-e^(-1))/2-s))
\end{verbatim}
\fi
\end{ex}

\subsection{Regra composta de Simpson}

A fim de obtermos a regra composta de Simpson, consideramos uma partição uniforme do intervalo de integração $[a, b]$ da forma $a=\tilde{x}_1 < \tilde{x}_2 < \cdots < \tilde{x}_{n+1}=b$, com $h=(\tilde{x}_{i+1}-\tilde{x}_{i})/2$, $i=1, 2, \dotsc, n$. Então, aplicando a regra de Simpson a cada integral nos subintervalos $[\tilde{x}_i, \tilde{x}_{i+1}]$, temos
\begin{align}
  \int_a^b f(x)\,dx &= \sum_{i=1}^{n}\int_{\tilde{x}_i}^{\tilde{x}_{i+1}}f(x)\,dx\\
  &= \sum_{i=1}^n \left\{\frac{h}{3}\left[f(\tilde{x_i}) + 4f\left(\frac{\tilde{x}_i+\tilde{x}_{i+1}}{2}\right) + f(\tilde{x_{i+1}})\right] + O(h^5)\right\}.
\end{align}
Então, observando que $h=(b-a)/(2n)$ e tomando $x_i=a+(i-1)h$, $i=1, 2, \dotsc, n$, obtemos a regra composta de Simpson\index{regra composta!de Simpson} com $n$ subintervalos
\begin{align}
  \int_a^b f(x)\,dx &= \frac{h}{3}\left[f(x_1) + 2\sum_{i=2}^{n} f(x_{2i-1}) + 4\sum_{i=1}^{n} f(x_{2i}) + f(x_{n+1})\right] \nonumber\\
  &+ O(h^4)
\end{align}

\begin{ex}\label{ex:int_comp_sim}
  Consideremos o problema de computar a integral de $f(x)=xe^{-x^2}$ no intervalo $[0, 1]$. Usando a regra composta de Simpson com $n$ subintervalos, obtemos a aproximação
  \begin{equation}
    \underbrace{\int_a^b f(x)\,dx}_{I} \approx \underbrace{\frac{h}{3}\left[f(x_1) + 2\sum_{i=2}^{n} f(x_{2i-1}) + 4\sum_{i=1}^{n} f(x_{2i}) + f(x_{n+1})\right]}_{S},
  \end{equation}
onde $h=1/(8n)$ e $x_i = (i-1)h$, $i=1, 2, \dotsc, n$. Na Tabela~\ref{tab:ex_int_comp_sim}, temos as aproximações computadas com diversos números de subintervalos, bem como, seus erros absolutos.

\begin{table}[h!]
  \centering
  \caption{Resultados referentes ao Exemplo~\ref{ex:int_comp_sim}.}
  \begin{tabular}{l|cc}
    $n$ & $S$ & $|I-S|$ \\\hline
    1   & $3,20914\E-1$ & $4,9\E-3$ \\
    10  & $3,16061\E-1$ & $3,4\E-7$ \\
    100 & $3,16060\E-1$ & $3,4\E-11$ \\
    1000& $3,16060\E-1$ & $4,2\E-15$ \\\hline
  \end{tabular}
  \label{tab:ex_int_comp_sim}
\end{table}

\ifisoctave
Podemos fazer estas computações com o auxílio do seguinte código \verb+GNU Octave+:
\begin{verbatim}
f = @(x) x*exp(-x^2);
a=0;
b=1;
n=1000;
h=(b-a)/(2*n);
s=f(a);
for i=2:n
  x=a+(2*i-2)*h;
  s+=2*f(x);
endfor
for i=1:n
  x=a+(2*i-1)*h;
  s+=4*f(x);
endfor
s+=f(b);
s*=h/3;
printf("%1.5E %1.1E\n",s,abs((1-e^(-1))/2-s))
\end{verbatim}
\fi
\end{ex}

\subsection*{Exercícios}

\begin{exer}\label{exer:int_comp_fun}
  Aproxime
  \begin{equation}
    \int_{-1}^0 \frac{\sen(x+2)-e^{-x^2}}{x^2+\ln(x+2)}\,dx
  \end{equation}
usando a:
\begin{enumerate}[a)]
\item regra composta do ponto médio com $10$ subintervalos.
\item regra composta do trapézio com $10$ subintervalos.
\item regra composta de Simpson com $10$ subintervalos.
\end{enumerate}
\end{exer}
\begin{resp}
  \ifisoctave 
  \href{https://github.com/phkonzen/notas/blob/master/src/MatematicaNumerica/cap_integr/dados/exer_int_comp_fun/exer_int_comp_fun.m}{Código.} 
  \fi
  a)~$2,69264\E-1$; b)~$2,68282\E-1$; c)~$2,68937\E-1$
\end{resp}

\begin{exer}\label{exer:int_comp_tab}
  Considere a seguinte tabela de pontos
  \begin{center}
    \begin{tabular}{l|cccccc}
      $i$ & $1$ & $2$ & $3$ & $4$ & $5$ & $6$ \\\hline
      $x_i$ & $2,0$ & $2,1$ & $2,2$ & $2,3$ & $2,4$ & $2,5$ \\
      $y_i$ & $1,86$ & $1,90$ & $2,01$ & $2,16$ & $2,23$ & $2,31$ \\\hline
    \end{tabular}
  \end{center}
Assumindo que $y = f(x)$, e usando o máximo de subintervalos possíveis, calcule:
\begin{enumerate}[a)]
\item $\displaystyle \int_{2,0}^{2,4} f(x)\,dx$ usando a regra do ponto médio composta.
\item $\displaystyle \int_{2,0}^{2,5} f(x)\,dx$ usando a regra do trapézio composta.
\item $\displaystyle \int_{2,0}^{2,4} f(x)\,dx$ usando a regra de Simpson composta.
\end{enumerate}
\end{exer}
\begin{resp}
  \ifisoctave 
  \href{https://github.com/phkonzen/notas/blob/master/src/MatematicaNumerica/cap_integr/dados/exer_int_comp_tab/exer_int_comp_tab.m}{Código.} 
  \fi
  a)~$8,12000\E-1$; b)~$1,03850$; c)~$8,11667\E-1$
\end{resp}

\section{Quadratura de Romberg}\label{cap_integr_sec_Romberg}

A quadratura de Romberg é construída por sucessivas extrapolações de Richardson da regra do trapézio composta. Sejam $h_k = (b-a)/(2k)$, $x_i = a + (i-1)h_k$ e
\begin{equation}
  R_{k,1} := \frac{h_k}{2}\left[f(a) + 2\sum_{i=2}^{2k}f(x_i) + f(b)\right]
\end{equation}
a regra do trapézio composta com $2k$ subintervalos de
\begin{equation}
  I := \int_a^b f(x)\,dx.
\end{equation}
Por sorte, o erro de truncamento de aproximar $I$ por $R_{k,1}$ tem a seguinte forma
\begin{equation}
  I - R_{k,1} = \sum_{i=1}^\infty k_ih_k^{2i},
\end{equation}
o que nos permite aplicar a extrapolação de Richardson para obter aproximações de mais alta ordem.

Mais precisamente, para obtermos uma aproximação de $I$ com erro de truncamento da ordem $h^{2n}$, $h=(b-a)$, computamos $R_{k,1}$ para $k=1, 2, \dotsc, n$. Então, usamos das sucessivas extrapolações de Richardson
\begin{equation}
  R_{k,j} := R_{k,j-1} + \frac{R_{k,j-1}-R_{k-1,j-1}}{4^{j-1}-1},
\end{equation}
$j=2, 3, \dotsc, n$, de forma a computarmos $R_{n,n}$, a qual fornece a aproximação desejada.

\begin{ex}\label{ex:Romberg}
  Consideremos o problema de aproximar a integral de $f(x)=xe^{-x^2}$ no intervalo $[0, 1]$. Para obtermos uma quadratura de Romberg de ordem $4$, calculamos
  \begin{align}
    R_{1,1} &:= \frac{1}{2}[f(0) + f(1)] = 1,83940\E-1\\
    R_{2,1} &:= \frac{1}{4}[f(0) + 2f(1/2) + f(1)] = 2,86670\E-1.
  \end{align}
Então, calculando
\begin{equation}
  R_{2,2} = R_{2,1} + \frac{R_{2,1}-R_{1,1}}{3} = 3,20914\E-1,
\end{equation}
a qual é a aproximação desejada.

\begin{table}[h!]
  \centering
  \caption{Resultados referentes ao Exemplo~\ref{ex:Romberg}.}
  \begin{tabular}{l|cccc}
    k & $R_{k,1}$ & $R_{k,2}$ & $R_{k,3}$ & $R_{k,4}$ \\\hline
    1 & $1,83940\E-1$ \\
    2 & $2,86670\E-1$ & $3,20914\E-1$ \\
    3 & $3,08883\E-1$ & $3,16287\E-1$ & $3,15978\E-1$ \\
    4 & $3,14276\E-1$ & $3,16074\E-1$ & $3,16059\E-1$ &  $3,16061\E-1$\\\hline
  \end{tabular}
  \label{tab:ex_Romberg}
\end{table}

Na Tabela~\ref{tab:ex_Romberg}, temos os valores de aproximações computadas pela quadratura de Romberg até ordem $8$.

\ifisoctave
Podemos fazer estas computações com o auxílio do seguinte código \verb+GNU Octave+:
\begin{verbatim}
#integral
f = @(x) x*exp(-x^2);
a=0;
b=1;

#ordem 2n
n=4;

R = zeros(n,n);
#R(k,1)
for k=1:n
  h = (b-a)/(2^(k-1));
  R(k,1) = f(a);
  for i=2:2^(k-1)
    x = a + (i-1)*h;
    R(k,1) += 2*f(x);
  endfor
  R(k,1) += f(b);
  R(k,1) *= h/2;
endfor
#extrapola
for j=2:n
  for k=j:n
    R(k,j) = R(k,j-1) + (R(k,j-1)-R(k-1,j-1))/(4^(j-1)-1);
  endfor
endfor
#sol.
for i = 1:n 
  printf("%1.5E ",R(i,:))
  printf("\n")
end
\end{verbatim}
\fi
\end{ex}

\subsection*{Exercícios}

\begin{exer}\label{exer:int_comp_fun}
  Aproxime
  \begin{equation}
    \int_{-1}^0 \frac{\sen(x+2)-e^{-x^2}}{x^2+\ln(x+2)}\,dx
  \end{equation}
usando a quadratura de Romberg de ordem 4.
\end{exer}
\begin{resp}
  \ifisoctave 
  \href{https://github.com/phkonzen/notas/blob/master/src/MatematicaNumerica/cap_integr/dados/exer_Romberg_fun/exer_Romberg_fun.m}{Código.} 
  \fi
  $2,68953\E-1$
\end{resp}

\emconstrucao
