%Este trabalho está licenciado sob a Licença Atribuição-CompartilhaIgual 4.0 Internacional Creative Commons. Para visualizar uma cópia desta licença, visite http://creativecommons.org/licenses/by-sa/4.0/deed.pt_BR ou mande uma carta para Creative Commons, PO Box 1866, Mountain View, CA 94042, USA.

\chapter{Equações Diferenciais Parciais}\label{cap_edp}
\thispagestyle{fancy}

Neste capítulo, discutimos alguns tópicos fundamentais da aplicação do método de diferenças finitas para a simulação (aproximação da solução) de equações diferenciais parciais.

\section{Equação de Poisson}\label{cap_edp_sec_poisson}\index{equação! de Poisson}

A equação de Poisson em um domínio retangular $D = (x_{\text{ini}}, x_{\text{fin}})\times (y_{\text{ini}}, y_{\text{fin}})$ com condições de contorno de Dirichlet homogêneas refere-se o seguinte problema
\begin{align}
  u_{xx} + u_{yy} &= f(x, y),~(x, y)\in D,\\
  u(x_{\text{ini}}, y) &= 0,~y_{\text{ini}}\leq y \leq y_{\text{fin}},\\
  u(x_{\text{fin}}, y) &= 0,~y_{\text{ini}}\leq y \leq y_{\text{fin}},\\
  u(x, y_{\text{ini}}) &= 0,~x_{\text{ini}}\leq x \leq x_{\text{fin}},\\
  u(x, y_{\text{fin}}) &= 0,,~x_{\text{ini}}\leq x \leq x_{\text{fin}},
\end{align}
onde $u = u(x,y)$ é a incógnita.

A aplicação do método de diferenças finitas para resolver este problema consiste dos mesmos passos usados para resolver problemas de valores de contorno (veja Capítulo~\ref{cap_pvc}), a saber: 1. contrução da malha, 2. discretização das equações, 3. resolução do problema discreto.

\begin{flushleft}
  {\bf 1. Construção da malha}
\end{flushleft}

Tratando-se do domínio retangular $\overline{D} = [x_{\text{ini}}, x_{\text{fin}}]\times [y_{\text{ini}}, y_{\text{fin}}]$, podemos contruir uma malha do produto carteziano de partições uniformes dos intervalos $[x_{\text{ini}}, x_{\text{fin}}]$ e $[y_{\text{ini}}, y_{\text{fin}}]$. Mais esplicitamente, tomamos
\begin{align}
  x_{i} &:= x_{\text{ini}} + (i-1)h_x,\quad h_x = \frac{x_{\text{fin}}-x_{\text{ini}}}{n_x-1},\\
  y_{j} &:= y_{\text{ini}} + (j-1)h_y,\quad h_y = \frac{y_{\text{fin}}-y_{\text{ini}}}{n_y-1},  
\end{align}
onde $i = 1, 2, \dotsc, n_x$, $j = 1, 2, \dotsc, n_y$, sendo $n_x$ e $n_y$ o número de subintervalos escolhidos para as partições em $x$ e $y$, respectivamente.

O produto cartesiano das partições em $x$ e $y$ nos fornece uma partição do domínio $\overline{D}$ da forma
\begin{equation}
  P(\overline{D}) = \{(x_1, y_1), (x_2, y_1), \dotsc, (x_i, y_j), \dotsc, (x_{n_x}, y_{n_y})\},
\end{equation}
cujos nodos $(x_i, y_j)$ podem ser indexados (enumerados) por $k = i + (j-1)n_x$.

\begin{flushleft}
  {\bf 2. Discretização das equações}
\end{flushleft}

\begin{flushleft}
  {\bf 3. Resolução do problema discreto}
\end{flushleft}

\emconstrucao

\section{Equação do calor}\label{cap_edp_sec_calor}\index{equação!do calor}

\emconstrucao

\section{Equação da onda}\label{cap_edp_sec_onda}\index{equação!da onda}

\emconstrucao