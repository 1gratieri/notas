%Este trabalho está licenciado sob a Licença Atribuição-CompartilhaIgual 4.0 Internacional Creative Commons. Para visualizar uma cópia desta licença, visite http://creativecommons.org/licenses/by-sa/4.0/deed.pt_BR ou mande uma carta para Creative Commons, PO Box 1866, Mountain View, CA 94042, USA.

\chapter{Técnicas de extrapolação}\label{cap_extrapl}
\thispagestyle{fancy}

Neste capítulo, estudamos algumas técnicas de extrapolação, as quais serão usadas nos próximos capítulos.

\section{Extrapolação de Richardson}\label{cap_extrapl_sec_Richardson}

Seja $F_1(h)$ uma aproximação de $I$ tal que
\begin{equation}\label{eq:extrapl_aux1}
  I = F_1(h) + \underbrace{k_1h + k_2h^2 + k_3h^3 + O(h^4)}_{\text{erro de truncamento}}.
\end{equation}
Então, divindo $h$ por $2$, obtemos
\begin{equation}\label{eq:extrapl_aux2}
  I = F_1\left(\frac{h}{2}\right) + k_1\frac{h}{2} + k_2\frac{h^2}{4} + k_3\frac{h^3}{8} + O(h^4).
\end{equation}
Agora, de forma a eliminarmos o termo de ordem $h$ das expressões acima, subtraimos $2$ vezes~\eqref{eq:extrapl_aux2} de \eqref{eq:extrapl_aux1}, o que nos leva a
\begin{equation}
  I = \left[F_1\left(\frac{h}{2}\right) + \left(F_1\left(\frac{h}{2}\right) - F_1(h)\right)\right] + k_2\frac{h^2}{2} + k_3\frac{3h^3}{4} + O(h^4).
\end{equation}
Ou seja, denotando
\begin{equation}
  N_2(h) := F_1\left(\frac{h}{2}\right) + \left(F_1\left(\frac{h}{2}\right) - F_1(h)\right)
\end{equation}
temos que $N_2(h)$ é uma aproximação de $I$ com erro de truncamento da ordem de $h^2$, uma ordem a mais de $N_1(h)$.



\emconstrucao