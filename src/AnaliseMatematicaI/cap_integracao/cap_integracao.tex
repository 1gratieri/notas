%Este trabalho está licenciado sob a Licença Atribuição-CompartilhaIgual 4.0 Internacional Creative Commons. Para visualizar uma cópia desta licença, visite http://creativecommons.org/licenses/by-sa/4.0/ ou mande uma carta para Creative Commons, PO Box 1866, Mountain View, CA 94042, USA.

\chapter{Integração}\label{cap_integracao}\index{integração}
\thispagestyle{fancy}

\section{Integral de Riemann}\label{cap_integracao_sec_riemann}\index{integral de!Riemann}

\begin{defn}\normalfont{(Partição de um intervalo)}\index{partição}
  Uma \emph{partição} $P$ de um intervalo $[a, b]$ é um conjunto ordenado da forma
  \begin{equation}
    P([a, b]) = \{a=x_0 < x_1 < x_2 < \ldots < x_n=b\}.
  \end{equation}
O valor $|P| = \max_{1\leq i\leq n} \Delta x_i$, $\Delta x_i = x_i - x_{i-1}$, é chamado de \emph{norma da partição}\index{norma da partição}.
\end{defn}

\begin{defn}\normalfont{(Refinamento de uma partição)}
  Um refinamento de uma partição $P_n([a, b]) := \{a=x_0, x_1, x_2, \dotsc, x_n=b\}$ é uma partição $P_m([a, b])$ com $m>n$ tal que $P_n([a, b])\subset P_m([a, b])$.
\end{defn}

\begin{defn}\normalfont{(Integral de Riemann)}
  A \emph{integral de Riemann}\index{integral de Riemann} de uma função $f:D\to\mathbb{R}$, $y=f(x)$, num intervalo $[a, b]\subset D$, quando existe, é o número $I$ tal que
  \begin{equation}
     I = \lim_{n\to \infty} \sum_{i=1}^n f(\xi_i)\Delta x_i,
  \end{equation}
onde arbitrariamente $\xi_i\in [x_{i-1}, x_{i}]$ e $\Delta x_i = x_{i}-x_{i-1}$ são tomados considerando todas as possíveis partições $P([a, b]) = \{a=x_0, x_1, x_2, \dotsc, x_n=b\}$, com $|P|\to 0$ quando $n\to 0$. Quando um tal $I$ existe, dizemos que $f$ é integrável em $[a, b]$.
\end{defn}

\begin{obs}
  As somas parciais
  \begin{equation}
    S_n = \sum_{i=1}^n f(\xi_i)\Delta x_i
  \end{equation}
que aparecem na definição da integral de Riemann são chamadas de \emph{somas de Riemann}\index{somas de!Riemann}.
\end{obs}

\section{Integrabilidade de funções contínuas}\label{cap_integracao_sec_intfc}

\begin{teo}\label{teo:integrabilidade_de_f_continua}
  Toda função $f:[a, b]\to\mathbb{R}$ contínua em $[a, b]$ é integrável.
\end{teo}
\begin{dem}
  Seja dado $\varepsilon>0$. Pelo teorema de Heine, $f$ é uniformemente contínua e, portanto, existe $\delta>0$ tal que
  \begin{equation}
    x,y\in I:=[a, b], |x-y|<\delta \Rightarrow |f(x)-f(y)|<\varepsilon.
  \end{equation}
Seja, agora, $S_n$ uma sequência arbitrária de somas de Riemann com norma tendo a zero quando $n\to\infty$, i.e.
\begin{equation}
  S_n = \sum_{i=1}^n f(\xi_i)\Delta x_i,
\end{equation}
com $\max_{1\leq i\leq n}\Delta x_i\to 0$ quando $n\to\infty$. Queremos provar que existe $I\in\mathbb{R}$ tal que
\begin{equation}
  I = \lim_{n\to\infty} S_n
\end{equation}
independentemente da escolha dos pontos $x_i$ e $\xi_i$. Para tanto, iremos usar o critério de convergência de Cauchy\index{Cauchy}. Para tanto, sejam 
\begin{equation}
  S_n := \sum_{i=1}^n f(\xi_i)\Delta x_i
\end{equation}
a soma de Riemann para uma dada partição $P_n := \{a=x_0<x_1<x_2<\cdots <x_n=b\}$ com $|P_n| < \delta$ e 
\begin{equation}
  S_M := \sum_{i=1}^M f(\eta_i)\Delta y_i
\end{equation}
a soma de Riemann para um refinamento $P_M:=\{a=y_0<y_1<y_2<\cdots <y_M=b\}$ de $P_n$. Como $P_M$ é um refinamento de $P_n$, cada subintervalo $[x_{i-1}, x_i]$ é a união de certos subintervalos $[y_{r-1}, y_{r}]$, $\ldots$, $[y_{s-1}, y_{s}]$ e, portanto $\Delta x_i = \Delta y_r + \cdots + \Delta y_s$. Ainda, a diferença $S_n - S_M$ conterá, então, termos da forma
\begin{equation}
  f(\xi_i)\sum_{j=r}^s f(\eta_i)\Delta y_j = \sum_{j=r}^s [f(\xi_i) - f(\eta_j)]\delta y_j.
\end{equation}
Agora, como $|\xi_i-\eta_j|<\delta$, temos $|f(\xi_i) - f(\eta_j)|<\varepsilon$ e, portanto
\begin{equation}
  \left|f(\xi_i) - \sum_{j=r}^s f(\eta_j)\Delta y_j\right| < \varepsilon \sum_{j=r}^s \Delta y_j = \varepsilon\Delta x_i.
\end{equation}
Estendendo este resultado, temos
\begin{equation}
  |S_n - S_M| \varepsilon\sum_{i=1}^n \Delta x_i = \varepsilon (b-a).
\end{equation}
Por fim, sejam $S_n$ e $S_m$ somas de Riemann correspondentes às partições $P_n$ e $P_m$, respectivamente, com $|P_n|<\delta$ e $|P_m|<\delta$. Ainda, seja $P_M$ um refinamento de ambas partições. Então
\begin{equation}
  |S_n - S_m| \leq |S_n - S_M| + |S_M - S_m| < 2\varepsilon (b-a).
\end{equation}
Isto mostra que, dada uma sequência arbitrária de partições $P_n$ com $|P_n|\to 0$ quando $n\to \infty$, então o limite das somas de Riemann $S_n$ destas partições existe quando $n\to\infty$. Falta mostrar que este limite é único.

Sejam, agora, $S_n$ e $T_n$ diferentes sequências de somas de Riemann cujas partições têm norma tendendo a zero quando $n\to\infty$. Então, por exemplo, a sequência
\begin{equation}
  S_1, T_1, S_2, T_2, S_3, T_3, \dotsc, S_n, T_n, \ldots
\end{equation}
também é uma sequência de somas de Riemann cujas partições têm norma tendo a zero quando $n\to\infty$. Logo, pelo que mostramos acima, o limite desta sequência existe. Agora, como $(S_n)_n$ e $(T_n)_n$ são subsequências destas, elas convergem para o mesmo limite.
\end{dem}

\subsection*{Exercícios}

\begin{exer}
  Mostre que se $f:[a, b]\to\mathbb{R}$ é integrável em $[a, b]$, então
  \begin{equation}
    \int_a^b f(x)\,dx = \int_a^c f(x)\,dx + \int_c^b f(x)\,dx,
  \end{equation}
para qualquer $c\in [a, b]$.
\end{exer}

\section{Teorema fundamental do cálculo}\label{cap_integracao_sec_tfc}

\begin{teo}\normalfont{(Teorema da média)}\index{teorema!da média}\label{teo:da_media}
  Sejam $f:[a,b]=:I\to\mathbb{R}$, $y=f(x)$, contínua em $I$ e $m$ e $M$ os valores mínimo e máximo de $f$ em $I$, respectivamente. Então, existe um número $c\in i$ tal que
  \begin{equation}
    \int_a^b f(x)\,dx = f(c)(b-a).
  \end{equation}
\end{teo}
\begin{dem}
  Observemos que toda a soma de Riemann satisfaz
  \begin{equation}
    m(b-a) \leq \sum_{i=1}^n f(\xi_i)\Delta x_i \leq M(b-a).
  \end{equation}
Passando ao limite quando $n\to \infty$, temos
\begin{equation}
    m(b-a) \leq \int_a^b f(x)\,dx \leq M(b-a),
\end{equation}
ou, equivalentemente, quando $a\neq b$,
\begin{equation}
    m \leq \frac{1}{b-a}\int_a^b f(x)\,dx \leq M.
\end{equation}
Agora, pelo teorema do valor intermediário (Teorema~\ref{teo:valor_intermediario}), existe $c\in I$ tal que
\begin{equation}
  f(c) = \frac{1}{b-a}\int_a^b f(x)\,dx.
\end{equation}
\end{dem}

\begin{teo}\normalfont{(Teorema fundamental do cálculo)}\index{teorema!fundamental do cálculo}\label{teo:teo_fundamental_do_calculo}
  Seja $f:[a, b]=:I\to\mathbb{R}$, $y=f(x)$, contínua em $I$. Então, a função $F:(a, b)\to\mathbb{R}$ definida por
  \begin{equation}
    F(x) = \int_a^x f(t)\,dt
  \end{equation}
é derivável em $(a, b)$ e $F'(x) = f(x)$.
\end{teo}
\begin{dem}
  Observemos que
  \begin{equation}
    F(x+h)-F(x) = \int_{a}^{x+h}f(t)\,dt - \int_{a}^x f(t)\,dt = \int_x^{x+h}f(t)\,dt.
  \end{equation}
Agora, do teorema da média (Teorema~\ref{teo:da_media}), existe $\xi \in [x, x+h]$ tal que
\begin{equation}
  \frac{F(x+h)-F(x)}{h} = f(\xi_{h}).
\end{equation}
Logo,
\begin{equation}
  F'(x) = \lim_{h\to 0}\frac{F(x+h)-F(x)}{h} = \lim_{h\to 0}f(\xi_{h}) = f(x).
\end{equation}
\end{dem}

\subsection*{Exercícios}

\begin{exer}
  Seja $f:[a, b]=:I\to\mathbb{R}$, $y=f(x)$, contínua em $I$. Então, a função $F:(a, b)\to\mathbb{R}$ definida por
  \begin{equation}
    F(x) = \int_a^x f(t)\,dt
  \end{equation}
é derivável à direta no ponto $a$ e à esquerda no ponto $b$ sendo, respectivamente, $F'(a+) = f(a)$ e $F'(b-) = f(b)$.
\end{exer}