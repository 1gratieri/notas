%Este trabalho está licenciado sob a Licença Atribuição-CompartilhaIgual 4.0 Internacional Creative Commons. Para visualizar uma cópia desta licença, visite http://creativecommons.org/licenses/by-sa/4.0/ ou mande uma carta para Creative Commons, PO Box 1866, Mountain View, CA 94042, USA.


%%%%%%%%%%%%%%%%%%%%%%%%%%%%%%%%%
%   Predefinicoes
%%%%%%%%%%%%%%%%%%%%%%%%%%%%%%%%%

\newif\ifisbook         % O layout será book?
\newif\ifishtml         % O layout será html?

\newif\ifisoctave       % Códigos em octave?

\def\tfn{config.knd}     % Arquivo que guarda as definições do tipo de saída
\def \tdata{}          % Definições do tipo de saída: book, slide ou html.

\openin1=\tfn\relax    % Leitura das definições de saída
\read1 to \tdata
\closein1

\tdata                 % Definições de saída

%%%%%%%%%%%%%%%%%%%%%%%%%%%%%%%%%
%   Opcões de Linguagem
%%%%%%%%%%%%%%%%%%%%%%%%%%%%%%%%%
\usepackage[brazil]{babel}
\usepackage[utf8]{inputenc}
\usepackage[T1]{fontenc}

\usepackage{lipsum}
\usepackage{fancyhdr}
\pagestyle{fancy}
\fancyhead{}
\renewcommand{\headrulewidth}{0pt}
\fancyfoot{}
\fancyfoot[RE,RO]{\thepage}
\fancyfoot[CE,CO]{\href{https://creativecommons.org/licenses/by-sa/4.0/deed.pt_BR}{Licença CC-BY-SA 4.0}}


%\pagestyle{fancy}
%\fancyhf{}
%\fancyhead[RE]{Análise Matemática}
%\fancyhead[LO]{\rightmark}
% \fancyhead[LE,RO]{\thepage}

%license footnote
%\cfoot{\tiny{Licença \href{https://creativecommons.org/licenses/by-sa/3.0/}{CC-BY-SA-3.0}. Contato: \url{reamat@ufrgs.br}}}

%%%% no blank pages between chapters %%%%
\let\cleardoublepage\clearpage

%%%% independent chapters %%%%
\usepackage{subfiles}

%%%% ams-latex %%%%
\usepackage{amsmath}
\usepackage{amssymb}
\usepackage{amsthm}
\usepackage{mathtools}

%%%% graphics %%%%
\usepackage{graphics}
\usepackage{graphicx}
%\usepackage{caption}

%%%% links %%%%
\usepackage[pdfborder={0 0 0 [0 0]},colorlinks=true,linkcolor=blue,citecolor=blue,filecolor=blue,urlcolor=blue]{hyperref}

%%%% copy and paste from PDF (correctly) %%%%
\usepackage{upquote}
\usepackage{lmodern}

%%%% code insert (verbatim) %%%%
\usepackage{verbatim}
\usepackage{listings}

%%%% indent first line %%%%
\usepackage{indentfirst}

%%%% comma as a decimal separator %%%%
\usepackage{icomma}

%%%% citation %%%%
\usepackage{cite}

%%%% lists %%%%
\usepackage{enumerate}

%%%% index %%%%
\usepackage{makeidx}


%%%% miscellaneous %%%%
\usepackage{multicol}
\usepackage{multirow}
\usepackage[normalem]{ulem}
\renewcommand{\arraystretch}{1.5} %space between rows in tables
\usepackage{array,booktabs}
\usepackage{xcolor}
\usepackage{tikz}


%%%%%%%%%%%%%%%%%%%%%%%%%%%%%%%%%%%%%%%%%%%%%%%%%%
% MACROS E NOVOS COMANDOS
%%%%%%%%%%%%%%%%%%%%%%%%%%%%%%%%%%%%%%%%%%%%%%%%%%
\newcommand*\circled[1]{\tikz[baseline=(char.base)]{
            \node[shape=circle,draw,inner sep=1pt] (char) {#1};}}
\newcommand{\RED}[1]{{\color{red}{#1}}}
\newcommand{\BLU}[1]{{\color{blue}{#1}}}
\newcommand{\GRE}[1]{{\color{darkgreen}{#1}}}
\newcommand{\matdd}[4]{\begin{bmatrix} #1&#2\\#3&#4 \end{bmatrix}}
\newcommand{\matddd}[9]{\begin{bmatrix} #1&#2&#3 \\ #4&#5&#6 \\ #7&#8&#9 \end{bmatrix}}
\newcommand{\vetdd}[2]{\begin{bmatrix} #1 \\#2 \end{bmatrix}}
\newcommand{\vetddd}[3]{\begin{bmatrix} #1 \\ #2\\ #3 \end{bmatrix}}
\newcommand{\field}[1]{\mathbb{#1}}
%emphasis \emph
\DeclareTextFontCommand{\emph}{\bfseries}
\newcommand{\sen}{\operatorname{sen}\,}
\newcommand{\senh}{\operatorname{senh}\,}
\renewcommand{\sin}{\operatorname{sen}\,}
\renewcommand{\sinh}{\operatorname{senh}\,}
\newcommand{\tg}{\operatorname{tg}\,}
\newcommand{\p}{\partial}
\newcommand{\Dom}{\operatorname{Dom}\,}
\newcommand{\diag}{\operatorname{diag}\,}

%E = 10^
\def\E#1{\mathrm{E}\!#1\!}

%%%%%%%%%%%%%%%%%%%%%%%%%%%%%%%%%%%%%%%%%%%%%%%%%%

\newcommand{\emconstrucao}{Em construção ...}

%%%%%%%%%%%%%%%%%%%%%%%%%%%%%%%%%%%%%%%%%%%%%%%%%%
  \theoremstyle{plain}          %   bold title, italic body
  \newtheorem{teo}{Teorema}[section]
  \newtheorem{lem}{Lema}[section]
  \newtheorem{prop}{Proposição}[section]
  \newtheorem{corol}{Corolário}[section]
  \newtheorem{defn}{Definição}[section]
  \theoremstyle{remark}           % italic title, romman body
  \theoremstyle{definition}       % italic title, romman body
  \newtheorem{obs}{Observação}[section]
  \newtheorem{ex}{Exemplo}[section]

%\renewenvironment{proof}{Demonstração.}

%\newenvironment{dem}{\bf Demonstração.}{\hfill $\blacksquare$\\} 

\newenvironment{dem}{\begin{proof}}{\end{proof}}


%Exercícios Resolvidos

\newtheorem{exeresol}{ER}[section]

%%%%%%%%%%%%%%%%%%%%%%%%%%%%%%%%%%%%%%%%%%%%%%%%%%
% + INTRUCOES PARA O FORMATO LIVRO
%%%%%%%%%%%%%%%%%%%%%%%%%%%%%%%%%%%%%%%%%%%%%%%%%%
\ifisbook
\input ../preambulo_book.tex
\fi
%%%%%%%%%%%%%%%%%%%%%%%%%%%%%%%%%%%%%%%%%%%%%%%%%%


%%%%%%%%%%%%%%%%%%%%%%%%%%%%%%%%%%%%%%%%%%%%%%%%%%
% + INTRUCOES PARA O FORMATO HTML
%%%%%%%%%%%%%%%%%%%%%%%%%%%%%%%%%%%%%%%%%%%%%%%%%%
\ifishtml
\input ../preambulo_html.tex
\fi
%%%%%%%%%%%%%%%%%%%%%%%%%%%%%%%%%%%%%%%%%%%%%%%%%%
