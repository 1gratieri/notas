%Este trabalho está licenciado sob a Licença Atribuição-CompartilhaIgual 4.0 Internacional Creative Commons. Para visualizar uma cópia desta licença, visite http://creativecommons.org/licenses/by-sa/4.0/ ou mande uma carta para Creative Commons, PO Box 1866, Mountain View, CA 94042, USA.

\chapter{Continuidade}\label{cap:continuidade}\index{continuidade}
\thispagestyle{fancy}

\section{Função contínua}\index{função!contínua}

\begin{defn}\normalfont{(Continuidade)}\label{defn:funcao_continua}
  Sejam $f:D\to\mathbb{R}$, $y=f(x)$, e $a$ um ponto de acumulação de $D$. Dizemos que $f$ é \emph{contínua} no ponto $a$ se as seguintes condições são satisfeitas:
  \begin{enumerate}[a)]
  \item $a\in D$.
  \item existe o limite de $f(x)$ com $x\to a$.
  \item $f(x)\to f(a)$ quando $x\to a$.
  \end{enumerate}
Ainda, dizemos que $f$ é uma \emph{função contínua}\index{função!contínua} (ou, simplesmente, contínua) quando $f$ é contínua em todos os pontos de seu domínio.
\end{defn}

\begin{ex}
  Vejamos os seguintes casos:
  \begin{enumerate}[a)]
  \item A função $f(x) = x-1$ é contínua em todo o seu domínio.
  \item A função $\displaystyle g(x)=\frac{x^2-1}{x+1}$ é \emph{descontínua}\index{função!descontínua} (i.e., não contínua) no ponto $x=-1$, pois este não é um ponto no domínio da função.
  \item A função
    \begin{equation}
      h(x) = \left\{\begin{array}{ll}\frac{x^2-1}{x+1} &, x\neq -1,\\1, x=-1\end{array}\right.
    \end{equation}
é descontínua no ponto $x=-1$, pois
    \begin{equation}
      \lim_{x\to -1} h(x) = -2 \neq 1 = h(-1).
    \end{equation}
  \end{enumerate}
\end{ex}

\begin{teo}
  Se $f$ e $g$ são funções contínuas no ponto $x=a$, então são contínuas nestes pontos as funções: (a) $f+g$, (b) $kf$, $\forall k\in\mathbb{R}$, (c) $f/g$, dado que $g(a)\neq 0$.
\end{teo}
\begin{dem}
  Decorre imediatamente da definição de função contínua (Definição~\ref{defn:funcao_continua}) e do Teorema~\ref{teo:operacoes_com_limites}.
\end{dem}

\begin{teo}\normalfont{(Continuidade da função composta)}
  Sejam dadas funções $f:D_f\to\mathbb{R}$ e $g:D_g\to\mathbb{R}$ com $g(D_g)\subset D_f$. Se $g$ é contínua no ponto $a$ e $f$ é contínua no ponto $g(a)$, então a função composta $f\circ g$ é contínua no ponto $a$.
\end{teo}
\begin{dem}
  É claro do enunciado que $a$ pertence ao domínio de $f\circ g$. Como $(f\circ g)(a) = f(g(a))$, nos resta mostrar que $(f\circ g)(x)$ tende para $f(g(a))$ quando $x\to a$. Seja, então, $\varepsilon>0$. Pela continuidade da $f$ no ponto $g(a)$, tomemos $\delta'>0$ tal que $y\in V'_{\delta'}(g(a))\cap D_f$ implica $|f(y)-f(g(a))|<\varepsilon$. Agora, pela continuidade da $g$ no ponto $a$, tomemos $\delta>0$ tal que $x\in V'_{\delta}(a)\cap D_g$ implica $|g(x)-g(a)|<\delta'$. Logo, temos que se $x\in V'_{\delta}(a)\cap D_g$, então $|f(g(x))-f(g(a))|<\varepsilon$, o que completa a demonstração.
\end{dem}

\begin{defn}\normalfont{(Continuidade lateral)}\index{continuidade!lateral}
  Dizemos que $f$ é \emph{contínua à direta}\index{função contínua!à direta} (\emph{contínua à esquerda}\index{função contínua!à direta}) no ponto $a$, se está definida neste ponto, onde seu limite à direta (à esquerda) é $f(a)$.
\end{defn}

\begin{ex}
  Vejamos os seguintes casos:
  \begin{enumerate}[a)]
    \item A função
      \begin{equation}
        f_1(x) = \left\{
          \begin{array}{ll}
            x/|x| &, x\neq 0,\\
            -1 &, x=0
          \end{array}
\right.
      \end{equation}
é contínua à esquerda no ponto $x=0$. De fato, $f_1(0)=-1$ e dado qualquer $\epsilon>0$ podemos escolher, por exemplo, $\delta = \epsilon$ tal que $0<0-x<\delta$ implica $|f_1(x)-(-1)|=|-1-(-1)|=0<\epsilon$.
    \item A função
      \begin{equation}
        f_2(x) = \left\{
          \begin{array}{ll}
            x/|x| &, x\neq 0,\\
            1 &, x=0
          \end{array}
\right.
      \end{equation}
é contínua à direta no ponto $x=0$. Verifique!
  \end{enumerate}
\end{ex}

\subsection*{Exercícios}

\begin{exer}
  Mostre que se $f:D\to\mathbb{R}$ é uma função contínua no ponto $a$ e $f(a)>0$, então existe $\delta>0$ tal que $x\in V_\delta(a)\cap D$ implica $f(x)>0$. Além disso, se removermos a hipótese de que $f$ seja contínua no ponto $a$ essa afirmação continua verdadeira? Justifique sua resposta.
\end{exer}
\begin{resp}
  Segue imediatamente do Corolário~\ref{corol:limite_permanencia_do_sinal}.
\end{resp}

\begin{exer}
  Mostre que qualquer $f:D\to\mathbb{R}$ é contínua em no ponto $a$ se, e somente se, $f$ é contínua à esquerda e à direita neste ponto.
\end{exer}
\begin{resp}
  Observe que $f$ tem limite no ponto $a$ se, e somente se, são iguais os limites à esquerda e à direita de $f$ neste ponto.
\end{resp}

\section{Propriedades de funções contínuas}

\begin{teo}\normalfont{Teorema do valor intermediário}\label{teo:valor_intermediario}\index{Teorema do!valor intermediário}
  Seja $f:D\to\mathbb{R}$ uma função contínua no intervalo fechado $I=[a, b]\subset D$, com $f(a)\neq f(b)$. Então, dado qualquer $d$ compreendido entre $f(a)$ e $f(b)$ (inclusive), existe $c\in I$ tal que $f(c)=d$.
\end{teo}
\begin{dem}
  \begin{enumerate}
  \item Primeiramente, notamos que o resultado é imediato para os casos de $d=f(a)$ e de $d=f(b)$.
  \item Suponhamos $f(a) < 0 < f(b)$ e, mostraremos que se $d=0$, então existe $c\in (a, b)$ tal que $f(c)=d$. Para tanto, usaremos o método da bisseção\index{método da!bisseção}. Seja $I^{(1)}:=[a^{(1)}, b^{(1)}]=[a, b]$, $l^{(1)}$ o comprimento do intervalo $I^{(1)}$ e $p^{(1)}$ o ponto médio deste. Se $f(p^{(1)})=0$ temos demonstrado o que queríamos. Agora, se $f(p^{(1)})>0$, escolhemos $I^{(2)}=[a, p^{(1)}]$. Entretanto, se $f(p^{(2)}) < 0$, escolhemos $I^{(2)} = [p^{(1)}, b]$. Em qualquer um dos casos $I^{(2)}:=[a^{(2)}, b^{(2)}]\subset I^{(1)}$, $l^{(2)} = l^{(1)}/2$ e $f(a^{(2)}) < 0 < f(b^{(2)})$. Com isso, repetimos o procedimento de bisseção para o intervalo $I^{(2)}$ com $p^{(2)}$ o ponto médio deste. Se $f(p^{(2)})=0$ temos o resultado desejado, caso contrário escolhemos o intervalo fechado $I^{(3)}:=[a^{(3)}, b^{(3)}]\subset I^{(2)}$, $l^{(3)}=l^{(2)}/2$ e $f(a^{(3)}) < 0 < f(b^{(3)})$. No pior dos casos, repetimos infinitamente este procedimento e, com isso, temos construído uma sequência de intervalos fechados $I^{(1)}\subset I^{(2)}\subset I^{(3)} \subset \cdots \subset I^{(n)} \subset \cdots$ cujos comprimentos tendem a zero. Logo, pelo Teorema dos intervalos encaixados\index{teorema!dos intervalos encaixados} $I^{(1)}\cap I^{(2)}\cap I^{(3)}\cap \cdots \cap I^{(n)}\cap \cdots = \{c\} \subset I$, o qual é o limite da sequência $a^{(n)}$ e da $b^{(n)}$. Daí, da continuidade da $f$ e do fato de $f(a^{(n)}) < 0 < f(b^{(n)})$ temos
  \begin{equation}
    0 \geq \lim f(a^{(n)}) = f(c) = \lim f(b^{(n)}) \geq 0,
  \end{equation}
donde segue que $f(c)=0$, como queríamos demonstrar.
  \item Suponhamos que $f(a) < f(b)$ e $d\in (f(a), f(b))$. Neste caso, tomamos $g(x) = f(x)-d$ e, portanto, temos $g(a) < 0 < g(b)$. Pelo demonstrado no item 2., existe $c\in [a, b]$ tal que $g(c)=0$ e, por consequência, $f(c)=d$.
  \item No caso de $f(a) > f(b)$, tomamos $g(x) = -f(x)$, de forma que $g(a) < g(b)$. Então, pelo item 3., temos o resultado desejado.
  \end{enumerate}
\end{dem}

\begin{lem}\label{lem:funcao_continua_limitada}
  Toda função contínua $f:I=[a, b]\to \mathbb{R}$ é limitada.
\end{lem}
\begin{dem}
  Demonstraremos por absurdo. Seja $f:I=[a, b]\to\mathbb{R}$ uma função contínua não limitada. Denotemos $I^{(1)} := I$. Como $f$ é não limitada em $I^{(1)}$, temos que $f$ é não limitada em pelo menos uma das metades do intervalo $I^{(1)}$. Seja, então $I^{(2)}$ uma das metades de $I^{(1)}$ na qual $f$ é não limitada. Sucessivamente, construímos uma sequência de intervalos fechados $I^{(n)}$ nos quais $f$ é ilimitada e cujos comprimentos tendem a zero. Então, pelo Teorema dos intervalos encaixados\index{teorema! dos intervalos encaixados}, existe um $c\in I^{(1)}\cap I^{(2)}\cap I^{(3)}\cap \cdots \cap I^{(n)}\cap \cdots \subset I$. Agora, pela continuidade de $f$, temos que $f(x)\to f(c)$ quando $x\to c$ e, portanto, existe $\delta>0$ tal que $x\in V_\delta(c)$ implica $f(c)-1 < f(x) < f(c)+1$, i.e. $f$ é limitada no intervalo $(c-\delta, c+\delta)$. Mas, como $I^{(n)}\subset (c-\delta, c+\delta)$ para $n$ suficientemente grande, temos $f$ limitada em $I^{(n)}$, o que é um absurdo.
\end{dem}

\begin{teo}\label{teo:funcao_continua_valores_extremos}
  Toda função contínua $f:I=[a, b]\to\mathbb{R}$ tem valor máximo e mínimo.
\end{teo}
\begin{dem}
  Vamos, primeiramente, mostrar que $f$ tem valor máximo em $I$. Por absurdo, seja $f:I=[a, b]\to\mathbb{R}$ função contínua, $M$ seu supremo (pelo Lema~\ref{lem:funcao_continua_limitada}, $f(I)$ é um conjunto limitado) e $f(x)<M$ para todo $x\in I$. Então, $1/(M-f(x))$ é uma função positiva e contínua em $I$. Seja, então, $M'>0$ seu supremo (novamente garantido pelo Lema~\ref{lem:funcao_continua_limitada}) e, então, para todo $x\in I$ temos
  \begin{equation}
    \frac{1}{M-f(x)}\leq M' \Rightarrow f(x) \leq M - \frac{1}{M'},
  \end{equation}
o que é um absurdo, pois isto contradiz o fato de $M$ ser o supremo de $f(I)$. Logo, existe algum $x\in I$ tal que $f(x)=M$. Analogamente, seja $m$ o ínfimo de $f(I)$. Então, $-m$ é o supremo da função $g(x) = -f(x)$ no intervalo $I$. Pelo que acabamos de demonstrar, existe $x\in I$ tal que $g(x)=-m$ e, por consequência, $f(x) = m$.
\end{dem}

\begin{teo}
  Se $f:I=[a, b]\to\mathbb{R}$ é uma função contínua, então $f(I)$ é um intervalo limitado e fechado.
\end{teo}
\begin{dem}
  Do Teorema~\ref{teo:funcao_continua_valores_extremos} sejam $m$ e $M$ os valores mínimo e máximo de $f$, respectivamente. Logo, $f(I) \subset [m, M]$. Agora, sejam $c,d\in I$ tal que $f(c)=m$ e $f(d)=M$. Pelo Teorema do valor intermediário\index{teorema!do valor intermediario}, dado qualquer $d\in [m, M]$ existe $x\in I$ tal que $f(x)=d$, i.e. $d\in f(I)$. Portanto, $[m, M]\subset f(I)$.
\end{dem}

\subsection*{Exercícios}

\begin{exer}
  Prove que todo o polinômio de grau ímpar $p(x) = a_nx^{n} + a_{n-1}x^{n-1} + \cdots + a_0$ tem no mínimo uma raiz.
\end{exer}
\begin{resp}
  Use o Teorema do valor intermediário\index{teorema!do valor intermediario}.
\end{resp}

\begin{exer}
  Dê um exemplo de:
  \begin{enumerate}[a)]
  \item uma função contínua não limitada $f:I\to\mathbb{R}$ com $I$ um intervalo limitado.
  \item uma função contínua $f:I\to\mathbb{R}$ definida em um intervalo ilimitado $I$ no qual $f$ tem valores mínimo e máximo.
  \end{enumerate}
\end{exer}

\section{Continuidade uniforme}\label{sec:continuidade_uniforme}\index{continuidade!uniforme}

\begin{defn}\normalfont{(Continuidade uniforme)}
  Uma função $f:D\to\mathbb{R}$, $y=f(x)$, é dita ser uniformemente contínua se, dado qualquer $\varepsilon>0$ existe $\delta>0$ tal que
  \begin{equation}
    x,y\in D, |x-y|<\delta \Rightarrow |f(x)-f(y)|<\varepsilon.
  \end{equation}
\end{defn}

\begin{ex}
  Vejamos os seguintes casos:
  \begin{enumerate}[a)]
  \item $f(x) = \sqrt{x}$ é uniformemente contínua. De fato, consideremos $x,y>0$ e $|x-y|<\delta$ para $\delta>0$ arbitrário. Então, se $y<\delta$ temos $x < y+\delta < 2\delta$ e
    \begin{equation}
      |\sqrt{x}-\sqrt{y}| \leq \sqrt{x} + \sqrt{y} < \sqrt{2\delta} + \sqrt{\delta} < 3\sqrt{\delta}.
    \end{equation}
Agora, se $y>=\delta$, então
\begin{equation}
  |\sqrt{x}-\sqrt{y}| = \frac{|x-y|}{\sqrt{x}+\sqrt{y}} < \frac{\delta}{\sqrt{y}} < \frac{\delta}{\sqrt{\delta}} = \sqrt{\delta}.
\end{equation}
Logo, em qualquer um dos casos $|\sqrt{x}-\sqrt{y}| < 3\sqrt{\delta}$. Por tanto, dado $\varepsilon > 0$, podemos escolher $\delta = \varepsilon^2/9$ de forma que
\begin{equation}
  x,y>0, |x-y|<\delta \Rightarrow |\sqrt{x}-\sqrt{y}| < \epsilon,
\end{equation}
o que conclui o resultado.
\item A função $f(x) = 1/x$ não é uniformemente contínua. De fato, basta observar que, para qualquer escolha de $\delta>0$, temos
  \begin{equation}
    \left|\frac{1}{x} - \frac{1}{x+\delta}\right| = \left|\frac{\delta}{x^2 + \delta x}\right|\to +\infty\quad\text{com}\quad x\to 0.
  \end{equation}
  \end{enumerate}
\end{ex}

\begin{teo}\normalfont{(de Heine)}
  Se $f:[a, b]\to\mathbb{R}$ é contínua em $[a, b]=:I$, então $f$ é uniformemente contínua.
\end{teo}
\begin{dem}
  Suponhamos, por contradição, que $f$ não é uniformemente contínua. Então, para algum $\epsilon>0$ existem $x_n,y_n\in I$ tal que
  \begin{equation}
    |x_n - y_n| < \frac{1}{n}\quad\text{e}\quad|f(x_n)-f(y_n)|>\epsilon,
  \end{equation}
para todo $n\in\mathbb{N}$. Agora, como $(x_n)_n$ é uma sequência limitada, pelo teorema de Bolzano-Weierstrass\index{teorema de!Bolzano-Weierstrass} ela possui uma subsequência convergente. Seja, então, $(x_{n'})_{n'}$ uma tal subsequência e $c$ o seu limite. Como $x_{n'}\in [a, b]$ para todo $n'$, temos $c\in [a, b]$. Além disso, como $|x_{n'}-y_{n'}|\to 0$, temos $y_{n'}\to c$. Também, pela continuidade de $f$, temos $f(x_{n'})\to f(c)$ e $f(y_{n'})\to f(c)$. Logo, $|f(x_{n'})-f(y_{n'})|\to 0$, o que é um absurdo.
\end{dem}

\subsection*{Exercícios}

\begin{exer}
  Mostre que se $f$ é uniformemente contínua em $(a, b)$, então existem os limites $\lim_{x\to a+} f(x)$ e $\lim_{x\to b^-}f(x)$.
\end{exer}
\begin{resp}
  Dica: 1) mostre que toda sequência $x_n\in (a, b)$ com $x_n\to a$ é tal que $f(x_n)$ é converge. Seja $L$ o limite desta sequência; 2) mostre, então, que qualquer outra sequência $y_n\in (a, b)$ com $y_n\to a$ é tal que $f(y_n)\to L$.
\end{resp}

