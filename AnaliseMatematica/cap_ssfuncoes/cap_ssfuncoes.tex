%Este trabalho está licenciado sob a Licença Atribuição-CompartilhaIgual 4.0 Internacional Creative Commons. Para visualizar uma cópia desta licença, visite http://creativecommons.org/licenses/by-sa/4.0/ ou mande uma carta para Creative Commons, PO Box 1866, Mountain View, CA 94042, USA.

\chapter{Sequências e séries de funções}\label{cap:sequências_e_séries_de_funções}\index{sequência de funções}\index{séries de funções}
\thispagestyle{fancy}

\section{Sequência de funções}\index{sequência de funções}

\begin{defn}
  Uma sequência de funções $(f_n)_{n\in\mathbb{N}}$ é um conjunto de funções $f_n:D\to\mathbb{R}$, $y=f_n(x)$, indexadas por $n\in\mathbb{R}$. Comumente, utiliza-se a notação $(f_n(x))_{n\in\mathbb{N}}$ (ou, simplesmente, $f_n(x)$) para explicitar que trata-se de uma sequência de funções.
\end{defn}

\begin{ex}
  Vejamos os seguintes exemplos:
  \begin{enumerate}[a)]
  \item $f_n(x) = x+1/n$, $n\in\mathbb{R}$, é uma sequência de funções afins.
  \item $g_n(x) = x^n$ é uma sequência de polinômios.
  \item $h_n(x) = 1 + x + x^2 + \cdots + x^n$ é, também, uma sequência\footnote{Um sequência deste tipo também é chamada de série de funções, como definiremos logo adiante no texto.} de polinômios.
  \end{enumerate}
\end{ex}

\subsection{Limite pontual}\index{limite!pontual}

\begin{defn}\normalfont{Limite pontual}
  Diz-se que uma sequência de funções $(f_n(x))_{n\in\mathbb{R}}$, $f_n:D\to\mathbb{R}$, \emph{converge pontualmente}\index{convergência!pontual} (ou simplesmente\index{convergência!simples}) para uma função $f(x)$, $f:D\to\mathbb{R}$, se, dado qualquer $\varepsilon>0$, para cada $x\in D$, existe $N$ tal que
  \begin{equation}
    n>N \Rightarrow |f_n(x)-f(x)|<\varepsilon.
  \end{equation}
\end{defn}

\begin{ex}
  Vejamos os seguintes casos.
  \begin{enumerate}[a)]
  \item A sequência de funções $f_n(x) = x + 1/n$ converge pontualmente para a função identidade $f(x)=x$. De fato, sejam $\varepsilon>0$ e $x$ no domínio da $f$. Escolhendo $N > 1/\varepsilon$, temos
    \begin{equation}
      n>N \Rightarrow |f_n(x) - f(x)| = \left|x+\frac{1}{n} - x\right| = \left|\frac{1}{n}\right| < \frac{1}{N} < \varepsilon.
    \end{equation}
  \item A sequência de funções $g_n(x) = x/n$ converge pontualmente para a função nula $f(x) \equiv 0$. De fato, sejam $\varepsilon>0$ e $x$ no domínio da $f$. Escolhendo $N > |x|/\varepsilon$, temos
    \begin{equation}
      n>N \Rightarrow \left|\frac{x}{n} - 0\right| < \frac{|x|}{n} < \frac{|x|}{N} < \varepsilon.
    \end{equation}
  \end{enumerate}
\end{ex}

\subsection*{Exercícios}

\begin{exer}
  Mostre que a sequência de funções $f_n:\mathbb{R}\setminus \{0\}$, $f_n(x) = 1/(nx)$, converge pontualmente para a função nula $f(x) \equiv 0$.
\end{exer}

\begin{exer}
  Mostre que a sequência de funções $f_n(x) = cos(x/n)$ converge pontualmente para função constante $f(x)\equiv 1$.
\end{exer}