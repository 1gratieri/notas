%Este trabalho está licenciado sob a Licença Atribuição-CompartilhaIgual 4.0 Internacional Creative Commons. Para visualizar uma cópia desta licença, visite http://creativecommons.org/licenses/by-sa/4.0/ ou mande uma carta para Creative Commons, PO Box 1866, Mountain View, CA 94042, USA.

\chapter{Fundamentos da análise}\label{cap:fundamentos_da_analise}\index{fundamentos da análise}
\thispagestyle{fancy}

\section{Funções}\label{sec:função}\index{função}

\subsection{Definição de função}

\begin{defn}\normalfont{(Função)}\index{definição de!função}
  Uma \emph{função}\index{função} $f:D \to Y$ é uma relação que associa cada elemento de um dado conjunto $D$ com um único elemento de um dado conjunto $Y$. O conjunto $D$ é chamado de \emph{domínio}\index{domínio} da função e o conjunto $Y$ é chamado de \emph{contradomínio}\index{contradomínio} da função.
\end{defn}

Comumente, uma dada função $f:D\to Y$ é acompanhada de sua \emph{lei de correspondência}\index{lei de correspondência}, a qual muitas vezes é denotada por $y = f(x)$. Neste caso, temos que a função $f$ associa $x\in D$ ao elemento $y\in Y$. Neste contexto, $x$ é chamada de \emph{variável independente}\index{variável!independente} e $y$ de \emph{variável dependente}\index{variável!dependente}. Ainda, muitas vezes uma função é descrita apenas por sua lei de correspondência e, neste caso, os conjuntos domínio e imagem são inferidos no contexto em questão.

\begin{obs}
  Neste livro, quando não especificado ao contrário, assumiremos que o domínio e o contradomínio das funções consideradas são subconjuntos dos números reais,
\end{obs}

\begin{ex} Vejamos os seguintes casos:
  \begin{enumerate}[a)]
  \item A relação $f:\{1,2,3\}\to\mathbb{R}$, $y = f(x) := x^2 + 1$, define uma função.
  \item A relação $g:D=\{0,1,2,3,4\}\to\mathbb{Z}$, $x^2 + y^2 = 9$ com $x\in D$ e $y\in Y$, não é uma função. Com efeito, $0\in D$ e relaciona-se com $3\in Y$ e $-3\in Y$ no seu contradomímio.
  \item Da equação $y = \sqrt{x}$ pode-se inferir a função $h:x\in D \to y\in\mathbb{R}$, onde o domínio $D$ é conjunto dos reais não negativos.
  \end{enumerate}
\end{ex}

\begin{defn}\normalfont{(Imagem de uma função)}
  A \emph{imagem}\index{imagem de!uma função} $I_f$ de uma dada função $f:D\to Y$ é o conjunto de todos os elementos de $Y$ que se relacionam com algum elemento de $D$, i.e.:
  \begin{equation}
    I_f := \{y\in Y;~\exists x\in D\text{ tal que }y=f(x)\}.
  \end{equation}
\end{defn}

\begin{ex}
  Vejamos os seguintes casos:
  \begin{enumerate}[a)]
  \item A função $f:\{1,2,3\}\to\mathbb{R}$, $y = f(x) := x^2 + 1$, tem imagem $I_f = \{1,4,9\}$.
  \item A imagem da função $f:\{0\}\cup\mathbb{N}\to\mathbb{R}$, $y = 2x+1$, é conjunto dos números ímpares.
  \item A imagem da função $\sen:\mathbb{R}\to\mathbb{R}$, $y=\sen x$, é $I_{\sen} = [-1, 1]$.
  \end{enumerate}
\end{ex}

\begin{obs}
  Dada uma função $f:D\to Y$ e um conjunto $A\subset D$, definimos a imagem de $A$ pela função $f$ por
  \begin{equation}
    f(A) := \{y\in Y;~\exists x\in A\text{ tal que }y=f(x)\}.
  \end{equation}
Por exemplo, dada a função $f:\mathbb{R}\to\mathbb{R}$, $y=\sqrt{x}$, temos
\begin{equation}
  f\left(\{0,1,4,9\}\right) = \{0,1,2,3\}.
\end{equation}
\end{obs}

\begin{defn}\normalfont{(Gráfico)} O \emph{gráfico}\index{gráfico} de uma função $f:D\to Y$, $y=f(x)$, é o conjunto de todos os pares ordenados $(x,y)$ tal que $x\in D$ e $y=f(x)$, i.e.
\begin{equation}
  G_f := \{(x, y)\in D\times Y;~y=f(x)\}.
\end{equation}
\end{defn}

\begin{ex}
  O gráfico da função $f:\{1,2,3\}\to\mathbb{R}$, $y = f(x) := x^2 + 1$, é
  \begin{equation}
    G_f = \{(1, 2), (2, 5), (3, 10)\}.
  \end{equation}
\end{ex}

\subsection{Classificações elementares}

\begin{defn}\normalfont{(Função limitada)}
  Seja dada uma função $f:D\to\mathbb{R}$, $y=f(x)$. Dizemos que $f$ é uma \emph{função limitada inferiormente}\index{função limitada!inferiormente} (ou \emph{limitada à esquerda}\index{função limitada!à esquerda}) quando existe $m\in \mathbb{R}$ tal que $m\leq f(x)$ para todo $x\in D$. Analogamente, dizemos que $f$ é uma \emph{função limitada superiormente}\index{função limitada!superiormente} (ou \emph{limitada à direta}\index{função limitada!à direita}) quando existe $M\in \mathbb{R}$ tal que $f(x)\geq M$ para todo $x\in D$. Ainda, $f$ é dita ser \emph{limitada}\index{função limitada} quando é limitada inferiormente e superiormente.
\end{defn}

\begin{ex}
  Vejamos os seguintes casos:
  \begin{enumerate}[a)]
  \item A função $f:\mathbb{R}\to\mathbb{R}$, $y=x^2+1$, é limitada inferiormente. De fato, para cada $x\in \mathbb{R}$ temos $x^2\geq 0$ e, portanto, $y = x^2+1 \geq 1$.
  \item A função seno é uma função limitada. Isto segue imediatamente da definição da função seno no círculo unitário (círculo trigonométrico).
  \end{enumerate}
\end{ex}

\begin{defn}\normalfont{Restrição/extensão de uma função}
  Uma função $g:A\to Y$, $y=g(x)$, é dita ser uma \emph{restrição}\index{restrição!de uma função} da dada função $f:D\to Y$ quando $A\subset D$ e $g(x)=f(x)$ para todo $x\in A$. Analogamente, $f$ é uma \emph{extensão}\index{extensão!de uma função} da função $g$.
\end{defn}

\begin{ex}
  A função $f:\mathbb{R}\to\mathbb{R}$, $y=x+1$, é uma extensão da função $g:\mathbb{R}\setminus\{1\}\to\mathbb{R}$, $\displaystyle y=\frac{x^2-1}{x-1}$.
\end{ex}

\begin{defn}\normalfont{(Função injetiva)}
  Uma função $f:D\to Y$, $y=f(x)$, é dita ser \emph{injetiva}\index{função!injetiva} (\emph{injetora} ou \emph{inversível}) quando para todo $x_1, x_2\in D$ com $x_1\neq x_2$ temos $f(x_1)\neq f(x_2)$.
\end{defn}

\begin{obs}
  Uma função $f:D\to\mathbb{R}$, $y=f(x)$, é injetiva se, e somente se, para todo $x_1, x_2\in D$ tal que $f(x_1)=f(x_2)$ temos $x_1=x_2$.
\end{obs}

\begin{ex}
  Vejamos os seguintes casos:
  \begin{enumerate}[a)]
  \item A função $f(x) = x^2$ não é injetiva, pois tomando $x_1=-1$ e $x_2=1$ temos $x_1\neq x_2$, mas $f(x_1)=f(x_2)$.
  \item A função $f(x) = \sqrt{x+1}$ é injetiva. De fato, dados $x_1, x_2\in\mathbb{D}$ tal que $f(x_1)=f(x_2)$, então $\sqrt{x_1} = \sqrt{x_2}$. Agora, tomando o quadrado dos dois lados, temos $x_1 = x_2$.
  \end{enumerate}
\end{ex}

\begin{defn}\normalfont{(Função sobrejetiva)}
  Uma função $f:D\to Y$, $y=f(x)$, é \emph{sobrejetiva}\index{função sobrejetiva} quando $f(D) = Y$ (ou, equivalentemente, $I_f = Y$).
\end{defn}

\begin{ex}
  A função $f:(0, \infty)\to\mathbb{R}$, $f(x) = \ln(x)$, é sobrejetiva. De fato, dado qualquer $y\in\mathbb{R}$ basta escolhermos $x = e^y$ para termos $f(x) = y$.
\end{ex}

\begin{obs}
  Uma função injetiva e sobrejetiva é dita ser \emph{bijetiva}\index{função!bijetiva}.
\end{obs}

\begin{defn}\normalfont{(Função inversa)}
  Dada uma função invertível (i.e. injetora) $f:D\to Y$, $y=f(x)$, definimos sua \emph{inversa}\index{função!inversa} por $f^{-1}:f(D)\to D$ que associa cada elemento $y\in f(D)$ com $x\in D$ tal que $f(x) = y$. 
\end{defn}

\begin{ex}
  Vejamos os seguintes casos:
  \begin{enumerate}[a)]
  \item A inversa da função $f:(0, \infty)\to \mathbb{R}$, $y = \ln(x)$, é a função $f^{-1}:\mathbb{R}\to (0, \infty)$, $y = e^{x}$.
  \item A inversa da função $f:[-1, \infty]\to [0, \infty)$, $y = \sqrt{x+1}$, é a função $f^{-1}:[0, \infty)\to [-1, \infty]$, $y = x^2 -1$. De fato, $f$ é sobrejetiva e dado $x\in [-1, \infty]$ temos $f(x) = y = \sqrt{x+1}$ e, então $y^2 = x + 1$, logo $x = y^2 - 1$.
  \end{enumerate}
\end{ex}

\begin{defn}\normalfont{(Função monótona)}
  Seja dada uma função $f:D\to Y$. Dizemos que $f$ é \emph{crescente}\index{função!crescente} quando para todo $x_1, x_2\in D$ com $x_1 < x_2$, temos $f(x_1) < f(x_2)$. Agora, quando para todo $x_1, x_2 \in D$ com $x_1 < x_2$ temos $f(x_1) \leq f(x_2)$, dizemos que $f$ é uma \emph{função não-decrescente}\index{função!não-decrescente}. Analogamente, quando para todo $x_1, x_2 \in D$ com $x_1 < x_2$ temos $f(x_1) > f(x_2)$ dizemos que $f$ é uma função \emph{decrescente}\index{função!decrescente}. Por fim, quando para todo $x_1, x_2 \in D$ com $x_1 < x_2$ temos $f(x_1) \geq f(x_2)$ dizemos que $f$ é uma função \emph{não-crescente}\index{função!decrescente}.
\end{defn}

\begin{ex}
  Vejamos os seguintes casos:
  \begin{enumerate}[a)]
  \item $f:\mathbb{R}\to\mathbb{R}$, $y = x^3$, é uma função crescente.
  \item $f:\mathbb{R}\to\mathbb{R}$, $y = e^{-x}$ é uma função decrescente.
  \end{enumerate}
\end{ex}

\begin{defn}\normalfont{(Paridade de uma função)}
  Uma função $f:D\to Y$, $y = f(x)$, é dita ser \emph{par}\index{função par} quando para todo $x\in D$, temos $f(x) = f(-x)$. Agora, quando para todo $x\in D$, temos $f(x) = -f(-x)$, então dizemos se tratar de uma função \emph{ímpar}\index{função!ímpar}.
\end{defn}

\begin{ex}
  Vejamos os seguintes casos:
  \begin{enumerate}[a)]
  \item A função $f:\mathbb{R}\to\mathbb{R}$, $y=|x|$, é uma função par.
  \item A função $f:\mathbb{R}\to\mathbb{R}$, $y=x^3$, é uma função ímpar.
  \end{enumerate}
\end{ex}

\subsection{Operações elementares}

Operações elementares envolvendo funções são comumente definidas tomando o cuidado de restringir o domínio das funções operadas para um conjunto apropriado. Por exemplo, dadas as funções $f:A\to \mathbb{R}$, $y=f(x)$, e $g:B\to\mathbb{R}$, $y=g(x)$, definimos a função soma de $f$ com $g$ por $(f+g):A\cap B\to\mathbb{R}$, $(f+g)(x) := f(x) + g(x)$. Agora, para estas mesmas função, definimos a função quociente de $f$ com $g$ por $(f/g):A\cap B\setminus \{0\}\to\mathbb{R}$, $(f/g)(x) := f(x)/g(x)$.

\begin{ex}
  A função $f:[0, \infty]\to\mathbb{R}$, $y = \sqrt{x} - |x|$, é a subtração da função $f_1:[0, \infty]\to\mathbb{R}$, $y = \sqrt{x}$, com a função $f_2:\mathbb{R}\to\mathbb{R}$, $y = |x|$, i.e. $f(x) = (f_1-f_2)(x):= f_1(x) - f_2(x)$.
\end{ex}

\begin{defn}\normalfont{(Composição de funções)}
  Sejam dadas as funções $f:D_f\to Y_f$, $y = f(x)$, e $g:D_g\to Y_g$, $y = g(x)$, com $I_g\subset D_f$. Definição a \emph{função composta}\index{função!composta} de $f$ com $g$ por $(f\circ g):D_g\to Y_f$ com $(f\circ g)(x) = f\left(g(x)\right)$.
\end{defn}

\begin{ex}
  A função $f:[0, \infty]\to\mathbb{R}$, $y=\sqrt{x^2 + 1}$, é a composição da função $f_1:[0, \infty]\to\mathbb{R}$, $y = \sqrt{x}$, com a função $f_2:\mathbb{R}\to\mathbb{R}$, $y = x^2+1$.
\end{ex}

\subsection*{Exercícios}

\begin{exer}
  Sejam $f:D\to Y$, $y=f(x)$, e $A, B \subset D$. Mostre que $f(A\cup B) = f(A)\cup f(B)$.
\end{exer}

\begin{exer}
  Construa uma função crescente, limitada superiormente e com domínio igual ao conjunto dos números reais.
\end{exer}

\begin{exer}
  Mostre que $f:[1, \infty)\to \mathbb{R}$, $y = \sqrt{x^3 - 1}$, é injetora e construa sua inversa.
\end{exer}

\begin{exer}
  Mostre que se $f:D\to Y$ é injetora, então $f$ não é par.
\end{exer}

\begin{exer}
  Mostre que uma dada função $f:\mathbb{R}\to\mathbb{R}$, $y = f(x)$, é limitada quando existe $c\in\mathbb{R}$ tal que $|f(x)|<c$, $\forall x\in\mathbb{R}$.
\end{exer}
